  \begin{titlepage}
    \begin{center}
      {{\Large{\textsc{Alma Mater Studiorum $\cdot$ Universit\`a di Bologna}}}} \rule[0.1cm]{15.8cm}{0.1mm}
      \rule[0.5cm]{15.8cm}{0.6mm}
      {\small{\bf SCUOLA DI SCIENZE\\
      Corso di Laurea Triennale in Informatica }}
    \end{center}
    
    \vspace{15mm}
    
    \begin{center}
      {\LARGE{\bf A Framework for Analysis}}\\
      \vspace{3mm}
      {\LARGE{\bf and innovative services}}\\
	  \vspace{3mm}
      {\LARGE{\bf for Electrical Mobility}}\\
      \vspace{15mm} 
      {\large{\bf Tesi di Laurea in Laboratorio di Applicazioni Mobili}}
    \end{center}
    
    \vspace{25mm}
    \par
    \noindent
    
    \begin{minipage}[t]{0.60\textwidth}
      {\large{\bf Relatore:\\
      Chiar.mo Prof.\\
      Luciano Bononi\newline}}\\
	{\large{\bf Correlatori:\\
	Chiar.mo Prof. Tullio S. Cinotti\\
	Dott. Marco Di Felice\\
	Dott. Luca Bedogni}}
    \end{minipage}
      \hfill
    \begin{minipage}[t]{0.34\textwidth}\raggedleft
    {\large{\bf Presentata da:\\
    Simone Rondelli}}  
    \end{minipage}
   \vspace{13mm}
    \begin{center}
	{\large{\bf Sessione III\\
	Anno Accademico 2012/2013}}
	\end{center}
	\clearpage{\pagestyle{empty}\cleardoublepage}%non numera l'ultima pagina sinistra
  \end{titlepage}
  
\cleardoublepage 
{\markboth{}{}
~\vspace*{20mm}
\begin{flushright}
\itshape
``Abbiamo la testa rotonda\\per pensare in tutte le direzioni.''\\
 Francis Picabia - (1922) 
\end{flushright}
\cleardoublepage 
}
  
\begin{abstract}

Internet of Energy is a European research project born with the aim of developing hardware and software infrastructure to introduce the electrical mobility in modern urban environments.



È stato oggetto di tesi di Federco Montori il quale ha sviluppato un primo prototipo di piattaforma comprendente un servizio cittadino di gestione delle ricariche, un'applicazione mobile che vi interagiva e infine un simulatore necessario al test della piattaforma. Nel corso di oltre un anno di sviluppo ho riscritto tutte le componenti software che costituivano il progetto ampliandone notevolmente le funzionalità, rendendole modulari e ben ingegnerizzate. Del progetto originario è stata ereditata l'architettura ontology-based basata sullo scambio di informazioni tramite il Semantic Information Broker (SIB).

Il mio contributo è iniziato nel 2012 con la riscrittura dell'applicazione mobile che inizialmente funzionava solo in presenza del simulatore. Attualmente permette di interfacciarsi a un veicolo reale tramite la tecnologia \emph{Blue\&{}Me} di Fiat. Questo approccio è stato reso possibile grazie all'opportunità offerta dal \emph{Centro Ricerche Fiat}, che ci ha permesso di testare presso loro sede l'applicazione mobile su un prototipo di Daily elettrico. Ho inoltre introdotto lo studio del profilo altimetrico e consumo energetico che separa il possessore dello smartphone da una determinata destinazione.

Nel 2013 ho deciso di riscrivere il Servizio Cittadino per renderlo conforme a un nuovo protocollo di prenotazione. Ho colto l'occasione per rendere il servizio altamente performante grazie a tecniche quali: pool di thread, pool di oggetti e caching.

Infine a cavallo tra il 2013 e il 2014 ho riscritto il simulatore al fine di ottimizzare il consumo di risorse, velocizzare il setup delle simulazioni e sopratutto renderlo più conforme alla realtà.

Questo lavoro ha permesso di avere una piattaforma software che permette di valutare realisticamente gli scenari di mobilità elettrica veicolare.

\end{abstract}