  \begin{titlepage}
    \begin{center}
      {{\Large{\textsc{Alma Mater Studiorum $\cdot$ Universit\`a di Bologna}}}} \rule[0.1cm]{15.8cm}{0.1mm}
      \rule[0.5cm]{15.8cm}{0.6mm}
      {\small{\bf SCUOLA DI SCIENZE\\
      Corso di Laurea Triennale in Informatica }}
    \end{center}
    
    \vspace{15mm}
    
    \begin{center}
      {\LARGE{\bf A Framework for Analysis}}\\
      \vspace{3mm}
      {\LARGE{\bf and innovative services}}\\
	  \vspace{3mm}
      {\LARGE{\bf for Electrical Mobility}}\\
      \vspace{15mm} 
      {\large{\bf Tesi di Laurea in Laboratorio di Applicazioni Mobili}}
    \end{center}
    
    \vspace{25mm}
    \par
    \noindent
    
    \begin{minipage}[t]{0.60\textwidth}
      {\large{\bf Relatore:\\
      Chiar.mo Prof.\\
      Luciano Bononi\newline}}\\
	{\large{\bf Correlatori:\\
	Chiar.mo Prof. Tullio S. Cinotti\\
	Dott. Marco Di Felice\\
	Dott. Luca Bedogni}}
    \end{minipage}
      \hfill
    \begin{minipage}[t]{0.34\textwidth}\raggedleft
    {\large{\bf Presentata da:\\
    Simone Rondelli}}  
    \end{minipage}
   \vspace{13mm}
    \begin{center}
	{\large{\bf Sessione III\\
	Anno Accademico 2012/2013}}
	\end{center}
	\clearpage{\pagestyle{empty}\cleardoublepage}%non numera l'ultima pagina sinistra
  \end{titlepage}
  
\begin{abstract}

Internet of Energy is a European research project born with the aim of developing hardware and software infrastructure to introduce the electrical mobility in modern urban environments.

It was Montori Federico's thesis \cite{montori2012} subject in 2012. He developed a first prototype of platform that included a City Service(CS) for manage the recharge reservations, a mobile application that can interact with the CS and a simulator for testing the platform. In the course of more than a year of development i rewrote all the software components of this project extending considerably the functionality, making it modular and better engineered. 

Inherited from the original project was the ontology-based architecture that relies on storing and exchanging information through the Smart-M3 framework. The ontological approach ensures interoperability between software agents who know the semantic structure of the exchanged data. All the entities that make up the domain model are defined by a class of ontology \cite{tullio2011}.

M3 is a middleware architecture born with the aim to share semantic information about the physical world in cross-domain, multi-vendor, multi-platform, multi-device, applications \cite{smart2010}. Smart-M3 is the first Open Source implementation of M3, proposed by SOFIA, a European Project (2009-11) of the ARTEMIS framework \cite{smart2013}. The heart of this architecture is the Semantic Information Broker (SIB) which has the task of storing and sharing information in M3 applications. The information is stored in RDF format.

All the information exchanged by the various agents of the system (cars, charging stations, city service etc\dots) pass through the SIB \cite{bedogni2013interoperable}. The communication protocols are implemented through the mechanism of subscriptions, provided by the SIB, which permits to receive a notification when there are changes made in a given set of data. Therefor the sender of a message puts a set of RDF Triples into SIB and the recipient receives a notification with the URI of message.

My contribution started in 2012 with the rewriting of the mobile application that initially worked only in a simulated environment. Now the application can interface with a real vehicle through the Blue\&{}Me technology. This is possible because Fiat is a partner of the IoE project and gave us the opportunity to go in their research center in Turin (CRF - Centro Ricerche Fiat) to test the application with a prototype of electrical Iveco Daily. I have introduced many other features such as the estimation of requested energy to reach a destination based on the study of the altimetric profile. The path between the user and the destination is drawn on the map with different colors to highlight the downhills (green), the uphills (red) and the flat stretches, so the user can choose the best road for reach the destination. The altimetric profile and the exact path to the destination is obtained thanks to a library that I have developed for this purpose (UniboGeoTools). Moreover I have mapped all the classes of ontology in an equivalent Entity class in Java (like the ORM approach) and I have created a Controller class that provides the simple CRUD operations. This greatly simplifies the programming activity.

During 2013, after a redefinition of the reservation protocol, i have rewrote completely the City Service. First, I made it compatible with the new protocol that now provides more parameters to choose the best Charge Option. I took the opportunity to make the service high performance through techniques such as thread pooling, object pooling and data caching \cite{larcab2005}. 

The last part of which the project comprises is the Simulator whose architecture is based on following three components:

\begin{itemize}
	\item{SUMO}: Simulator of Urban MObility \cite{behrisch2011sumo} which allowed to model the scenario of Bologna. The SUMO simulations are purely microscopic: each vehicle is modeled explicitly, has an own route, and moves individually through the network. In fact you can keep control of each vehicle of the simulation (and many other parameters) through a TCP/IP interface called TraCI \cite{wegener2008traci}. 
	\item{OMNeT++}: a discrete event simulator born with the aim to build network simulations \cite{varga2001omnet++}. It allows to define many simulation aspects such as vehicle logic, the battery model, the charging station model and the drivers' behaviour.
	\item{Veins}: an OMNeT++ module which permits to take control of a vehicle in the SUMO simulation through TraCI interface \cite{sommer2011bidirectionally}\cite{kopke2008simulating}. For each vehicle in the traffic simulation Veins creates a corresponding module in OMNeT++.
\end{itemize}

The initial version of simulator was only a proof of concept and could not be used for collecting useful data. Therefor i have decided to rewrite it from scratch. I greatly improved the application design, performance and features. Now the simulator can be used to evaluate the impact of introducing the electrical mobility in a realistic scenario. The simulator keeps track of the occupation of charging stations , the various states of the vehicle (eg: going to recharge, charging, normal driving etc\dots), the number of failed tries to recharge and some information about each vehicle such as battery consumption, speed, altitude, acceleration etc\dots

The simulation can be executed with or without the reservation service, so you can measure the benefits of this service. The charging stations are configured by an external xml file that describes their power, voltage and current.

You can study the number of Electrical Vehicle supported by the city \cite{katmale2011} by making a simulation in which you can vary the number of vehicles and the penetration rate of Electrical Vehicle. Than for example is possible to study where would be better to place a new charging station or where it would be sensible to improve the efficiency of existent ones.

Another important feature that I have introduced in the simulator is the possibility to import the altimetric profile of the scenario in the simulation and the introduction of it in the battery consumption model. This is possible thanks to the SRTM files that enhance the map (imported by Open Street Map) that  constitutes the scenario. The consumption model takes into account the regenerative braking that permits the battery to recharge when the driver brakes \cite{dinicola2014}.

It is possible to vary a lot of parameters to customize the scenario, you can change the behavior of the driver that establishes, for example, what a driver does when arriving at charging station that is occupied, or which charging station he chooses when the battery is low. You can vary some vehicles parameters such as weight,  engine efficiency, battery capacity etc\dots

Ultimately I have developed a very useful service that can build the trust of users towards electric mobility and an efficient simulation framework to test it. The simulator can be used both by energy providers and by the  public administration to test the maximum number of electrical vehicles that the city can support, the impact of this upon traffic and many other important parameters.

The entire \LaTeX  sources of this thesis are available in italian at \url{https://github.com/monejava/internet-of-energy}

\end{abstract}