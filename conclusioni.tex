\chapter{Conclusioni}

Al termine di questa esperienza durata oltre un anno posso asserire di aver portato un grande contributo al progetto Internet of Energy su diversi fronti. Partendo da un progetto già esistente, ma in fase estremamente embrionale, ho creato una piattaforma solida e modulare. 

L'applicazione mobile è passata da essere uno strumento puramente dimostrativo a fornire la possibilità di interfacciarsi con veicoli reali.

Il Servizio Cittadino è stato ulteriormente migliorato ed ora è anche accessibile dall'esterno senza piattaforma simulativa in esecuzione.

Infine la piattaforma di simulazione ha raggiunto un notevole grado di realismo e stabilità che hanno portato ad aderire appieno ai requisiti del progetto IoE. 

Ritengo pertanto di aver contribuito al progetto e di aver svolto un lavoro che, spero, possa vedere la luce ed essere utilizzato nei prossimi anni all'interno di questi contesti.


\chapter{Ringraziamenti}

Desidero ringraziare il professori Luciano Bononi e Tullio Salmon Cinotti per aver creduto in me e avermi dato la possibilità di partecipare a questo progetto. È stata un esperienza lunga e impegnativa ma estremamente formante e per certi versi dipendente. Indimenticabili saranno gli innumerevoli meeting (brain-storming) che hanno determinato l'evoluzione del progetto.

Un grande ringraziamento va a Federico Montori, il \emph{papà} di questo progetto. In brevissimo tempo ha gettato le basi che hanno permesso a questo progetto crescere fino all'attuale stato di maturità.

Come non ringraziare Luca Bedogni e Marco Di Felice che sono stati sempre presenti. Grazie a loro ho imparato cosa significa fare simulazioni, le giornate passate nel Laboratorio di Reti Wireless sono state al contempo frustranti ed eccitanti. Innumerevoli le ore passate a tentare di raffinare i modelli di simulazione al fine di avvicinarli alla realtà.

Alfredo D'Elia come si farebbe senza di lui? È stato la \emph{spina dorsale} del progetto, l'interfaccia tra le idee, a volte eccessivamente visionarie, dei professori ed il sottoscritto che le doveva implementare. La sua capacità di problem solving, anche su problemi che non gli competono direttamente, mi ha sempre sbalordito.

Ringrazio la mia ragazza per avermi incentivato a iniziare a scrivere questa tesi, se non fosse stato per lei non avrei scritto nemmeno una pagina ma avrei continuato a \emph{giocare} con il simulatore.

Infine ringrazio la mia famiglia che mi ha supportato e ha creduto in me, chiudendo spesso un occhio sulla mia assenza in casa. E nondimeno ha pagato un anno in più di retta per permettermi di seguire i miei interessi.











