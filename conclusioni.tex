\chapter{Conclusioni}

Al termine di questa esperienza durata oltre un anno posso asserire di aver portate un grande contributo al progetto Internet of Energy su diversi fronti. Partendo da un progetto già esistente, ma in fase estremamente embrionale, ho creato una piattaforma solida e modulare. Il principio di base che ha guidato lo sviluppo è stato rendere l'architettura portabile e semplice da capire al fine di permettere a chiunque di partecipare attivamente al progetto. Difatti quattro sono stati i progetti di entità minore sviluppati in seno a questo progetto, personalmente ho seguito ognuno di essi, ed è stata una notevole soddisfazione vedere i risultati che ne sono conseguiti.

La modularità è stata ottenuta applicando i noti concetti di ingegneria del software che hanno portato ad avere un software più facile da capire, e quindi tramandare, e mantenere. 

Ho automatizzato diverse fasi del progetto, infatti bisogna considerare che quando ho preso a mano il progetto lo scenario simulabile era uno solo e discostarsi da esso risultava notevolmente difficile. Oggi in meno di mezz'ora si è in grado di creare uno scenario funzionante per uno scenario urbano completamente diverso da Bologna. 

In primo luogo l'applicazione mobile, aspetto che mi ha permesso di accedere al progetto, è cresciuta notevolmente... bla bla bla...

In secondo luogo il Servizio Cittadino in 


Infine la piattaforma di simulazione ha raggiunto un notevole grado di realisticità e stabilità che hanno portato ad aderire appieno ai requisiti del progetto IoE. La maturità e la genericità della piattaforma sviluppata è tale che ha attirato l'interesse per un altro grande progetto europeo, ARROWHEAD. 







