\documentclass[12pt,a4paper,openright,twoside]{report}

\usepackage[italian]{babel} %a capo italiano
\usepackage[T1]{fontenc}
%\usepackage[latin1]{inputenc}%libreria per accettare i caratteri si può usare anche \usepackage[T1]{fontenc}
\usepackage{fancyhdr}%libreria per impostare il documento
\usepackage{indentfirst}%libreria per avere l'indentazione
\usepackage{newlfont}
\usepackage[utf8]{inputenc} %fornisce i caratteri accentati
\usepackage{graphicx}
\usepackage{float}
\usepackage{enumitem} % ref enumerate
\usepackage{xcolor}
\usepackage{caption}
\usepackage{subcaption}
\usepackage[hyphens]{url}
\usepackage{hyperref}
\usepackage{underscore}
\usepackage[babel]{csquotes}
\usepackage[backend=biber]{biblatex} %biblio
\usepackage[export]{adjustbox} %ridimensionamento immagini
\usepackage{textcomp} %simbolo grado celsius
\usepackage{amsmath}

\defbibheading{cartaceo}{\subsection*{Manuali cartacei}}
\defbibheading{web}{\subsection*{Siti Web consultati}}

%%% Impostazioni pagine %%%
\usepackage{fancyhdr}
%\pagestyle{fancy}\addtolength{\headwidth}{20pt}
\renewcommand{\chaptermark}[1]{\markboth{\MakeUppercase{\chaptername}\ \thechapter.\ #1}{}}
\renewcommand{\chaptermark}[1]{\markboth{\thechapter.\ #1}{}}
\renewcommand{\sectionmark}[1]{\markright{\thesection \ #1}{}}
\rhead[\fancyplain{}{\bfseries\leftmark}]{\fancyplain{}{\bfseries\thepage}}
\cfoot{}

\newcommand{\clear}{\clearpage{\pagestyle{empty}\cleardoublepage}}

\setlength{\headheight}{15pt}


\addbibresource{biblio.bib}

\usepackage{listings} %box per codice sorgente
\usepackage{color} %colori
\usepackage{courier} %usa il curier in \ttfamily

% pacchetto che usa bergami!!
% permette ad esempio di usare \ttfamily (monospaced) nel codice e avere comunque alcune parole in grassetto (come loop nel codice arduino, ecc..)
\usepackage[scaled]{beramono}

\newcommand{\code}[1]{\texttt{#1}}

% per vedere ``Listato'' anziché ``Listing'' come didascalia sopra ai frammenti di codice
\addto\captionsitalian{\renewcommand{\lstlistingname}{Listato}}
% per vedere ``Listato'' anziché ``Listings'' come indice dei frammenti di codice
\addto\captionsitalian{\renewcommand{\lstlistlistingname}{Elenco dei Listati}}

%\DeclareCaptionFont{white}{\color{white}}
%\DeclareCaptionFormat{listing}{\colorbox[cmyk]{0.43, 0.35, 0.35,0.01}{\parbox{\dimexpr\textwidth-2\fboxsep\relax}{#1#2#3}}}
%\captionsetup[code]{format=listing,labelfont=white,textfont=white, singlelinecheck=false, margin=0pt, font={bf,footnotesize}}

\definecolor{my_gray}{RGB}{240,240,240}


\lstdefinelanguage{BASH} {
	otherkeywords={=, +, [, ], (, ), \{, \}, *},
    %morecomment=[l]{#},
    %morestring=[s]{"}{"},
    morekeywords={make,apt-get,sudo},
    breaklines=true,
    keywordstyle=\bfseries,
    stringstyle=\color{red},
    emphstyle=\color{black}\bfseries,
    commentstyle=\color{darkgreen}\slshape,
}

\lstnewenvironment{bash}[1][] {
	\lstset{language=BASH,
           basicstyle=\ttfamily\footnotesize,
           numbers=left,
           numberstyle=\tiny\color{gray},
           backgroundcolor=\color{lightgray},
           breaklines=true,
           breakatwhitespace=true,
           mathescape=false,
           morecomment=[l]{\#},
           stepnumber=1,
		   firstnumber=1,
           #1
          }
}{}

%XML
\definecolor{maroon}{rgb}{0.5,0,0}
\definecolor{darkgreen}{rgb}{0,0.5,0}
\lstdefinelanguage{XML} {
	basicstyle=\ttfamily,
	morestring=[s]{"}{"},
	morecomment=[s]{?}{?},
	morecomment=[s]{!--}{--},
	commentstyle=\color{darkgreen},
	moredelim=[s][\color{black}]{>}{<},
	moredelim=[s][\color{red}]{\ }{=},
	stringstyle=\color{blue},
	identifierstyle=\color{maroon}
}

\lstnewenvironment{xml}[1][] {
	\lstset{language=XML,
            basicstyle=\ttfamily\footnotesize,
            numbers=left,
            numberstyle=\tiny\color{gray},
            backgroundcolor=\color{lightgray},
            breaklines=true,
            breakatwhitespace=true,
            stepnumber=1,
			firstnumber=1,
			captionpos=bc,
           #1
          }
}{}

%%%%%%%%%%%%%%%%%%%%%%%%%% JAVA %%%%%%%%%%%%%%%%%%%%%%%%%%%%%%%%

\definecolor{javared}{rgb}{0.6,0,0} % for strings
\definecolor{javagreen}{rgb}{0.25,0.5,0.35} % comments
\definecolor{javapurple}{rgb}{0.5,0,0.35} % keywords
\definecolor{javadocblue}{rgb}{0.25,0.35,0.75} % javadoc

\lstnewenvironment{java}[1][] {
	\noindent\minipage{\linewidth}%
	\lstset{language=Java,
			basicstyle=\ttfamily\footnotesize,
			keywordstyle=\color{javapurple}\bfseries,
			stringstyle=\color{javared},
			commentstyle=\color{javagreen},
			morecomment=[s][\color{javadocblue}]{/**}{*/},
			showspaces=false,
			showstringspaces=false
            basicstyle=\ttfamily\footnotesize,
            numbers=left,
            tabsize=2,
            numberstyle=\tiny\color{black},
            backgroundcolor=\color{lightgray},
            breaklines=true,
            breakatwhitespace=true,
            stepnumber=1,
			firstnumber=1,
			captionpos=bc,
            #1
          }
	%\captionsetup{options=code}
}{\endminipage}

%http://jevopi.blogspot.it/2010/03/nicely-formatted-listings-in-latex-with.html

\oddsidemargin=30pt \evensidemargin=20pt%impostano i margini

%comandi per l'impostazione della pagina, 
\pagestyle{fancy}\addtolength{\headwidth}{20pt}
\renewcommand{\chaptermark}[1]{\markboth{\thechapter.\ #1}{}}
\renewcommand{\sectionmark}[1]{\markright{\thesection \ #1}{}}
\rhead[\fancyplain{}{\bfseries\leftmark}]{\fancyplain{}{\bfseries\thepage}}
\cfoot{}

% comando per impostare l'interlinea
\linespread{1.2}

%\textwidth=450pt\oddsidemargin=0pt
\begin{document}
	 % comando per evitare che il contenuto di una pagina venga ``stiracchiato'' se non c'è abbastanza
	 % testo lo spazio in bianco viene lasciato in fondo alla pagina e non qua e la
	\raggedbottom 
	
		
	  \begin{titlepage}
    \begin{center}
      {{\Large{\textsc{Alma Mater Studiorum $\cdot$ Universit\`a di Bologna}}}} \rule[0.1cm]{15.8cm}{0.1mm}
      \rule[0.5cm]{15.8cm}{0.6mm}
      {\small{\bf FACOLTÀ DI SCIENZE MATEMATICHE, FISICHE E NATURALI\\
      Corso di Laurea Triennale in Informatica }}
    \end{center}
    
    \vspace{15mm}
    
    \begin{center}
      {\LARGE{\bf IoE}}\\
      \vspace{3mm}
      {\LARGE{\bf Internet of Energy}}\\
      \vspace{19mm} 
      {\large{\bf Tesi di Laurea in Laboratorio di Applicazioni Mobili}}
    \end{center}
    
    \vspace{40mm}
    \par
    \noindent
    
    \begin{minipage}[t]{0.47\textwidth}
      {\large{\bf Relatore:\\
      Chiar.mo Prof.\\
      LUCIANO BONONI}}
    \end{minipage}
      \hfill
    \begin{minipage}[t]{0.47\textwidth}\raggedleft
    {\large{\bf Presentata da:\\
    SIMONE RONDELLI}}
    \end{minipage}
    \vspace{20mm}
    \begin{center}
      {\large{\bf Sessione (che cazzo ne so?)\\%inserire il numero della sessione in cui ci si laurea
      Anno Accademico }}%inserire l'anno accademico a cui si è iscritti
    \end{center}
  \end{titlepage}
  
\begin{abstract}
    Abstract \LaTeX.
    
    Abbiamo la testa rotonda 
	per pensare in tutte le direzioni. 
 
	Francis Picabia - (1922) 
\end{abstract}
	
	\clear
	
	\tableofcontents{}
	
	\clear
	  
	\chapter{Introduzione}

\section{Smart Cities}
Citta intelligenti

\section{Electrical Mobility}

Al giorno d'oggi l'Electrical Mobility (EM) è considerata una degli elementi chiave per ridurre l'inquinamento 
e al contempo liberarsi dalla dipendenza dai combustibili fossili. Questo sta portando a ingenti investimenti
da parte di governi e delle industrie automobilistiche. 

Nel breve periodo il mercato legato all'EM è destinato a crescere rapidamente come  
consuguenza dell'incremento della varietà di Veicoli Elettrici (EVs) introdotti dalle Case Automobilistice. 
Secondo recenti studi infatti il numero di EVs venduti nel periodo tra il 2010 e il 2012 è aumentato del 200\%.
Nonostante il crescente interesse nei confronti dell'EM, recenti analisi di mercato dimostrano
che i benefici ad essa legati saranno tangibili soltanto nel lungo questo è confermato da una ricerca condotta dal 
U.S. National Energy Technology secondo cui il 70\% delle persone non comprerà un EV a causa 
dell'incertezza sulla disponibilità delle stazioni di ricarica. A questo si vanno ad aggiungere le
ben note problematiche riguardanti la capacità, la durata delle batterie e i tempi di ricarica 
estremamente lunghi (nell'ordine delle decine di minuti).

Da un lato la durata dei tempi di ricarica, la limitata capacità delle batterie e la disposizione
degli Electric Vehicle Supply Element influsce direttamente sull'esperienza di guida di ogni autista 
e può avere un impatto decisivo sulla penetrazione di mercato dei veicoli elettrici.
D'altra parte diversi studi hanno dimostrato che l'impatto sulla rete energetica causato dalla
ricarica simultanea di molti Veicoli Elettrici può avere ripercussioni negative e si è 
quindi delineata la necessità di coordinare le attività tra EVs ed EVSEs 

Molti progetti Europei sono stati avviati con lo scopo di limitare queste problematiche. Allo stesso tempo
bisogna considerare che un uno scenario realistico di EM ci sono diverse parti interessate
(es: autisti, case automobilistiche, produttori di energia) coinvolte nella gestione dell'EM.
La ricerca si è mobilitata in direzione dell'Information and Communication Technology (ICT)
per fornire servizi di supporto all'EM e permettere alle parti interessate di cooperare in modo intelligente.
Sebbene siano state sviluppate diverse applicazioni su scenari in piccola scala, si è ancora lontani
dall'ottenere l'interoperabilità tra gli attori in gioco i quali utilizzano diverse tecnologie e dispositivi.

Dato l'elevato costo che avrebbero i test su larga scala, la simulazione costituisce lo strumento più adatto per testare l'efficenza delle suluzioni ICT prima che vengano realmente sviluppate. Al giorno d'oggi sono stati sviluppati alcuni simulatori veicolari che permettono un controllo molto fine a livello di veicolo e similarmante altrettanti modelli di batteria sono stati create al fine di riprodurre in modo realistico le dinamiche di carica e scarica della batteria.
Tuttavia nessuno di questo strumenti è adatto al fine di studiare le dinamiche assai complesse che si presentano 
nello scenario dell'EM, come l'impatto degli EV sulla rete elettrca cittadina oppure l'effettiva utilità dell'utilizzo di sistemi di prenotazione delle ricariche. 

Il progetto Internet of Energy (IoE) for Electrical Mobility, fondato dall'Unione Europea e comprendente 40 partner da 10 nazioni Europee, mira a colmare queste lacuna, sviluppando hardware, software e sistemi middleware che forniranno un infrastruttura di comunicazione interpolabile tra le parti in gioco all'interno della (? dire due parole a riguardo smart grid).

Lo scopo di questa tesi, frutto del lavoro congiunto tra UNIBO e ARCES, seguito della tesi di laurea di 
Federico Montori, è fornire contributi su tre diversi fronti a questo progetto. 

Inanzitutto abbiamo sviluppato un architettura software con lo scopo di fornire servizi per l'interazione
tra gi EVs ed EM attraverso lo smartphone. Il servizio centrale è il City Service (CS) il quale
si prende a carico le richieste di ricarica, che arrivano dagli smartphone, fornendo la lista degli EVSEs 
disponibili che piu si adattano alle esigenze dell'utente. Il modello di dati usato dal servizio si basa
su un ontologia che rappresenta tutte le informazioni relative alla smart-grid. Le informazioni sono condivise
attraverso un repository semantico chiamato Semantic Information Broker (SIB), esso garantisce, grazie all'ontologia,
un interazione uniforme tra i vari componenti del sistema.

In secondo luogo abbiamo creato un'applicazione mobile che permette all'utente di monitorare i parametri
della batteria del veicolo e di prenotare slot di tempo presso gli EVSE grazie all'interazione con la 
SIB cittadina. Il servizio di prenotazione da la possibilita di scegliere in base a vari parametri come il prezzo, 
la distanza, il contributo energetico necessario a raggiungere l'EVSE e il tempo totale di ricarica.

Infine abiamo creato una piattaforma di simulazione integrata che permette di valutare su larga scala
l'impatto della EM. Diversamente ad altri tool gia presenti il nostro framework permette di studiare 
il comportamento degli EV, con relativo modello di carica e scariaca della batteria, insieme all'interazione
di essi con la samrt grid attraverso gli EVSE. A questo proposito sono stati usati doversi tool tra i quali
SUMO, un ``simulatore di traffico microscopico``, OMNET++, un simulatore a eventi discreti, e infine per 
far comunicare i due simulatori viene usata l'interfaccia TRACI, messa a disposizione da SUMO, la quale
comunica con OMNET++ attraverso Veins. Grazie a SUMO riusciamo a modellare l'ambiante urbano, nel nostro 
caso Bologna e Torino, includendo dati topografici e altimetrici realistici. OMNET++ invece è stato
utilizzato per implementare i modelli dell'EV, compresa la batteria e il comportamento dell'autista.

\section{Internet of Energy}

\subsection{Artemis}

%partner importanti enel crf fiat ambito energia...%http://www.artemis-ioe.eu/

\subsection{Far progredire il mondo dell'energia}

%livelli: mobilità, ricarica, batterie, standard, colonnine, 

\subsection{Mobilità/software}

%colonnina
%business
%

\section{Lavori Correlati}



\section{Un po di storia} 

\textbf{DA LASCIARE??? a me \emph{MI} piace}

L'applicazione mobile è stato il "cavallo di troia" con il quale ho avuto il privilegio di partecipare a questo progetto. Era il 19/07/2012 quando ho inviato al Prof \emph{Luciano Bononi} la richiesta di progetto per il corso di \emph{Laboratorio di Applicazioni Mobili}. Non potevo immaginare che quella email mi avrebbe portato ad un lavoro durato oltre un anno e che dura tutt'ora.
Al colloquio per l'assegnazione del progetto mi venne presentato questa opportunità: ovvero rendere più accattivante un applicazione mobile fatta da Federico Montori per il suo progetto di laurea. Questo perché, visto il poco tempo che egli aveva potuto dedicarci, era ancora in fase embrionale.
Cosi in seguito a qualche meeting con i vari componenti del progetto e notevoli dosi di pazienza da parte di \emph{Federico Montori}, il quale mi ha introdotto al progetto, sono riuscito ad avere un ambiente più o meno funzionate, la simulazione mi crashava dopo un secondo ma tanto bastava per introdurre un veicolo e fare i test con la nuova applicazione mobile.
Successivamente, capendo la vastità del progetto che stava dietro a tale applicazione, decisi di farlo diventare progetto di laurea in quanto era evidente che c'era ancora molto lavoro da fare e la mia innata capacità per complicarmi la vita ha giocato un ruolo fondamentale in tutto questo.

??


	
	\clear
	
	\chapter{Architettura}

In questo capitolo verranno descritte le scelte architetturali e implementative che stanno alla base del contributo di questa tesi al progetto IoE. 

Lo scenario legato alla mobilità elettrica veicolare è caratterizzato dalla presenza di diversi domini applicativi, piattaforme e parti interessate i quali necessitano di comunicare in modo unificato e trasparente. A tal fine è stato utilizzato il progetto Smart-M3 (\cite{tullio2011}) che è il cuore della nostra architettura. Appoggiandosi sulle tecnologie tipiche del \emph{Semantic Web} Smart-M3 assicura l'interoperabilità tra gli attori in gioco. 

In particolare vedremo come possono coesistere elementi reali ed elementi simulati e come il passaggio dall'uno all'altro sia assolutamente trasparente a tutti i componenti del sistema grazie all'uso di tecnologie ontology-based.

\section{Smart-M3}

Prima di parlare dei componenti strettamente legati a questo progetto è doveroso fare un introduzione alla tecnologia che fa da collante tra di essi ovvero Smart-M3. Capire come funziona Smart-M3 e quali sono i suoi principi è fondamentale al fine di comprendere a fondo il resto di questo documento.

M3 è un architettura middleware per consentire L'interoperabilità delle informazioni in maniera cross-domain, multi-vendor, multi-device, multi-piattaforma (\cite{smart2013}). Smart-M3 è la sua prima implementazione Open Source, proposta da SOFIA, un Progetto Europeo (2009-11), appartenente al framework ARTEMIS. 
La piattaforma implementa il disaccoppiamento tra produttori e consumatori di informazione. In questa architettura tutti gli attori (sensori, dispositivi, servizi, attuatori ecc..) cooperano attraverso un database RDF che è lo standard deciso dal World Wide Web Consortium per la descrizione di informazioni e concetti. L'interoperabilità è resa possibile da un modello di dati condiviso che si basa su tecnologie tipiche del Semantic Web.

Il Semantic Web è un framework sviluppato dal World Wide Web Consortium per consentire la condivisione e il riutilizzo dei dati attraverso  applicazioni, aziende e comunità eterogenee.

La figura ~\ref{fig:smart-m3} mostra il funzionamento dell'architettura M3. Il "legacy gate" è un interfaccia con il mondo esterno e di essi ne possono coesistere innumerevoli in un architettura M3.

\begin{figure}[H]
	\centering
	\includegraphics[width=0.5\textwidth]{assets/smart-m3.jpg}
	\caption{Architettura Smart-M3}
	\label{fig:smart-m3}
\end{figure}

\subsection{Semantic Information Broker}\label{subsec:sib}

Il \emph{Semantic Information Broker} (SIB) è l'entità responsabile della conservazione e della gestione delle informazioni condivise nell'architettura M3. Gli agenti Software che si scambiano le informazioni vengono chiamati \emph{Knowledge Processors} (KPs). L'accesso alla SIB da parte dei KP avviene attraverso lo \emph{Smart Space Access Protocol}  (SSAP), esso consiste in messaggi XML scambiati attraverso socket TCP/IP. Vengono fornite API che implementano il protocollo SSAP in diversi linguaggi.

Il SIB è un architettura a 5 livelli (\cite{smart2010}) come mostrato in figura ~\ref{fig:sib-architecture}:

\begin{enumerate}
	\item \textbf{Transport}: Questo livello consiste in una o più comunicazioni di rete a livello di trasporto che permette al SIB di comunicare con diverse reti e architetture. Il livello di trasporto è collegato a quello sottostante tramite il DBus, rendendo possibile l'aggiunta e la rimozione di connettori a runtime.
	\item \label{enum:handling}\textbf{Operation Handling}: A questo livello vengono gestite le operazioni del protocollo SSAP e ognuna di esse viene eseguita in un thread dedicato. Malgrado l'uso intensivo di thread possa degradare le performance la chiarezza di codice che ne consegue è stata ritenuta più importante.
	\item \label{enum:graph}\textbf{Graph Operations}: Questo livello gestisce le operazioni di inserimento, rimozione e query dal database RDF come richiesto dal livello ~\ref{enum:handling}. Viene eseguito all'interno di un singolo thread che schedula ed esegue le richieste provenienti dai thread che gestiscono le operazioni SSAP la quale comunicazione avviene tramite code asincrone.
	\item \textbf{Triple Operations}: A questo livello vengono gestite le operazioni SPARQL, WQL e le query basate su pattern-matching di triple RDF. Attualmente è implementato tramite Piglet, un database RDF che si appoggia ad SQL lite per la persistenza delle informazioni. Questo strato può essere tranquillamente cambiato a patto che si scriva il codice necessario ad interfacciare le operazioni a livello di grafo (~\ref{enum:graph}) con l'interfaccia fornita dal nuovo store RDF.
	\item \textbf{Persistent storage}: Quetso è il livello che assicura la persistenza dei dati.
\end{enumerate}

\subsection{I Knowledge Processor}

I Knowledge Processor (KP) sono le parti attive dell'architettura Smart-M3. Un KP interagisce con il SIB non direttamente tramite il protocollo SSAP ma tramite le Knowledge Processor Interface (KPI) ovvero delle librerie che lo implementano. Esse possono trovarsi a qualunque livello di astrazione ed essere scritte in qualunque linguaggio. Le funzioni messe a disposizione dal KPI in genere sono speculari alle operazioni del protocollo SSAP.

I KP sono quelle entità che forniscono, modificano e richiedono le informazioni le informazioni contenute nello smart-space. L'architettura dei KP è mostrata in figura ~\ref{fig:kp-architecture}.

\begin{figure}[H]
        \centering
        \begin{subfigure}[H]{0.5\textwidth}
                \includegraphics[width=\textwidth]{assets/sib-architecture.jpg}
                \caption{Architettura del SIB}
                \label{fig:sib-architecture}
        \end{subfigure}%
        \begin{subfigure}[H]{0.42\textwidth}
                \includegraphics[width=\textwidth]{assets/kp-architecture.jpg}
                \caption{Architettura dei KP}
                \label{fig:kp-architecture}
        \end{subfigure}
        \caption{Architetture SIB e KP}
\end{figure}

\subsection{Le triple RDF}

Nell'architettura Smart-M3 le informazioni sono rappresentate in formato RDF (Resource Description Framework). In RDF le informazioni sono rappresentate come una tripletta \emph{soggetto, predicato, oggetto}. Le triple vengono memorizzate nel SIB e formano un grafo etichettato diretto il quale non necessariamente è un grafo connesso.

\subsection{Ontologie}

Mentre RDF fornisce il modello di dati standard per la rappresentazione delle informazioni, l'uso di un linguaggio ontologico è indispensabile per assegnare una semantica all'informazione. Linguaggi ontologici come RDFS e OWL forniscono un vocabolario comune. L'uso di una ontologia comune consente a tutti gli attori (uomini e macchine) di capire reciprocamente la semantica delle informazioni e di cooperare in simbiosi attraverso il SIB. Smart-M3 è agnostico rispetto all'ontologia e quindi consente agli sviluppatori di scegliere il modo migliore di modellare le informazioni al fine di soddisfare le esigenze funzionali del dominio applicativo indirizzato.

\subsection{Sottoscrizioni}

Un aspetto fondamentale di questa tecnologia è il meccanismo delle sottoscrizioni grazie al quale è possibile ricevere notifiche al variare di set di triple. Le sottoscrizioni sono determinanti nella nostra architettura perché, come vedremo più avanti (Sez. ~\ref{sec:protocol}), sono alla base dei protocolli di scambio dati tra i componenti del sistema.

\subsection{SPARQL}

SPARQL, si pronuncia sparkle, (acronimo ricorsivo di SPARQL Protocol and RDF Query Language) è il linguaggio standard de facto per interrogare dei dataset RDF. Come si può dedurre dal nome stesso SPARQL non è semplicemente un linguaggio di interrogazione di dati RDF, ma definisce anche il protocollo applicativo utilizzato per comunicare con le sorgenti RDF (si tratta di un binding su HTTP).

Così come SQL riflette, nella rappresentazione della query, il modello relazionale sottostante, allo stesso modo SPARQL basa la rappresentazione della query sul concetto di tripla e di grafo. Il meccanismo alla base della rappresentazione di una query e della ricerca della sua risposta è il graph matching. La query rappresenta un pattern di un grafo (RDF) e la risposta alla query sono tutte le triple (sotto-grafo) che fanno match con il pattern.

\subsection{Il protocollo SSAP}

L'SSAP (\emph{Smart Space Access Protocol}) è il protocollo con cui si comunica con il SIB. Il protocollo è session-based, i KP che vogliono comunicare con lo smart-space dovranno prima aderirvi con un operazione di Join prevista dall'SSAP. Il KP fornisce le sue credenziali nel messaggio di Join, il SIB esamina le credenziali e decide se accetare il KP o meno. Dopo l'operazione di Join, il KP può eseguire le altre operazioni. 
L'SSAP è il punto di integrazione principale dell'architettura Smart-M3. Le implementazioni di SIB e KP devono implementare tutte le operazioni del protocollo SSAP al fine di garantire l'interoperabilità.

Le operazioni supportate dal protocollo SSAP sono:

\begin{itemize}
	\item \textbf{JOIN}: Associa il KP allo smart-space solo se le credenziali verranno ritenute valide. Determina l'inizio della sessione.
	\item \textbf{LEAVE}: Determina il termine dell'associazione con lo smart-space e quindi la fine della sessione. Da questo momento in poi non potranno essere eseguite altre operazione di associazione allo smart-space.
	\item \textbf{INSERT}: Operazione atomica di inserzione di un Grafo, formato da triple RDF, nel SIB.
	\item \textbf{REMOVE}: Operazione atomica di rimozione di un Grafo, formato da triple RDF, nel SIB.
	\item \textbf{UPDATE}: Operazione atomica di aggiornamento di un Grafo, formato da triple RDF, nel SIB. In realtà si tratta di una combinazione di DELETE e INSERT eseguita in modo atomico dove l'operazione di DELETE viene eseguita per prima.\textsl{•}
	\item \textbf{QUERY}: Richiesta di informazioni contenute nel SIB attraverso una delle modalità supportate.
	\item \textbf{SUBSCRIBE}: Sottoscrizione a un set di triple contenute nel SIB. Il KP riceve una notifica quando avviene un cambiamento su una di queste triple.
	\item \textbf{UNSUBSCRIBE}: Cancella una sottoscrizione.
\end{itemize}

%\begin{figure}
%	\centering
%	\includegraphics[width=0.5\textwidth]{assets/smart-m3-domain.jpg}
%	\caption{Smart-M3 modello di dominio}
%	\label{fig:smart-m3-domain}
%\end{figure}

\section{Il Modello Ontologico}

In questa sezione verrà spiegato come sono stati modellati i dati attraverso un ontologia. Il modello ontologico è stato ereditato dalla progetto di Tesi di \emph{Federico Montori} (\cite{montori2012}). Io ho contribuito espandendolo al fine di adattarlo ai nuovi requisiti funzionali sorti durante lo sviluppo del progetto. Verranno quindi mostrate gli aspetti dell'ontologia necessari a comprendere il resto della trattazione e verranno approfondite le modifiche da me apportate.

\subsection{Introduzione}

L'ontologia è definibile come una rappresentazione formale ed esplicita di una concettualizzazione condivisa di un dominio di interesse.

L'ontologia presenta le seguenti proprietà:

\begin{itemize}
	\item \textbf{Rappresentazione Formale}: Utilizza pertanto un linguaggio logico processabile da elaboratori.
	\item \textbf{Esplicita}: Cioè non ambigua e tale da chiarire ogni assunzione fatta.
	\item \textbf{Concettuale}: È una concettualizzazione cioè una vista astratta e semplificata del dominio di interesse
	\item \textbf{Condivisa}: Determinata dal consenso di una pluralità il più ampia possibile di soggetti.
\end{itemize}

Lo scopo delle ontologie è quindi descrivere delle basi di conoscenze, effettuare delle deduzioni su di esse e integrarle tra le varie applicazioni. Per descrivere le ontologie viene utilizzato il linguaggio OWL (\emph{Ontology Web Language}) che è un estensione di RDF. È un linguaggio di markup per rappresentare esplicitamente significato e semantica di termini con vocabolari e relazioni tra gli stessi.

I linguaggi della famiglia OWL sono in grado di creare \emph{classe}, \emph{proprietà}, \emph{istanze} e le relative \emph{operazioni}.


\subsection{Classi di IoE}

Una classe è una collezione di oggetti che corrisponde alla descrizione logica di un concetto. Da una classe si possono creare un numero arbitrario di istanze mentre ad un istanza può corrispondere ad una, nessuna o molteplici classi.\newline
Una classe può essere sottoclasse di un'altra classe, ereditando le caratteristiche della super-classe. Tutte le classi sono sottoclasse di \code{owl:Thing}.

Nel modello di dati utilizzato in questo progetto si è cercato di tenere disaccoppiato il concetto di dato dalle altre entità fisiche. Ne consegue che tutte le entità fisiche sono sottoclassi dirette di \code{owl:Thing}, mentre le classi destinate a rappresentare i dati sono sottoclasse di \code{ioe:Data} che a sua volta è sottoclasse di \code{owl:Thing}.\newline
Nel resto di questo documento userò il prefisso \code{ioe:} come abbreviazione di \emph{http://www.m3.com/2012/05/m3/ioe-ontology.owl\#} che è il namespace scelto per l'ontologia. In generale userò questo prefisso per distinguere le classi dell'ontologia dalle classi Java che come vedremo nella sezione ~\ref{subsec:ioe-lib} hanno lo stesso nome essendo mapping diretto di quest'ultime.

\subsection{Sottoclassi di owl:Thing}

Come già accennato tutte le entità fisiche del nostro modello ontologico sono sottoclasse diretta di owl:Thing. Quellla mostrata di seguito è una lista delle Classi utilizzate in questo progetto, vengono omesse quelle che sono attualmente irrilevanti o inutilizzate.

\begin{itemize}
	\item \textbf{Person}: Questa classe rappresenta il concetto di persona. Ad ogni persona possono essere associati diversi veicoli (Vehicle), diverse richieste di ricarica (Reservation), nonché una storia delle ricariche effettuate (Recharge). Il concetto di persona viene utilizzato ai fini di autenticazione e in un futuro potrà essere determinante ai fini della fatturazione che il provider energetico eseguirà in seguito alle ricariche.
	\item \textbf{Vehicle}: Rappresenta il concetto di Veicolo Elettrico, siccome i veicoli non elettrici sono irrilevanti al fine di questa trattazione è stata usata direttamente questa classe allo scopo. Ad ogni veicolo sono ovviamente associati i dati della batteria (BatteryData) che verranno trattati nella sezione relativa alle sottoclassi di \code{ioe:Data} (~\ref{subsec:ioe-data});
	\item \textbf{GridConnectionPoint}: Il \emph{Grid Connection Pointer} (GCP) è la stazione di ricarica. Esso contiene almeno un EVSE che invece rappresenta la colonnina dove ci si ricarica effettivamente. Il rapporto tra un GCP e gli EVSE è lo stesso che intercorre tra una stazione di rifornimento e le pompe di benzina. 
	\item \textbf{EVSE}: Il \emph{Electrical Vehicle Supply Equipment} è il punto in cui il veicolo si connette alla rete elettrica. Una volta connesso può sia ricaricare la sua batteria che cedere energia alla smart-grid. Un EVSE ha diversi connettori (Connector) per adattarsi ai vari tipi di presa posseduti dai veicoli elettrici. Inoltre, ogni EVSE, ha una lista di prenotazioni associate.
	\item \textbf{ChargeProfile}: È l'insieme dei parametri che caratterizzano il profilo energetico di un EVSE in un determinato istante. I parametri attualmente sono: potenza, orario di validità del profilo stesso e prezzo per unità di energia (in genere 1 kWh). Ovviamente può essere attivo un solo ChargeProfile alla volta e il variare di quest'ultimi può essere determinato da fasce orarie proprio come avviene per l'energia elettrica casalinga.
	\item \textbf{Connector}: È il connettore di ricarica ovvero il punto i contatto tra l'EVSE e l'EV. Ogni EVSE può avere diversi connettori al fine di poter essere compatibile con il maggior numero di veicoli possibile. Malgrado negli usa si stia cercando di introdurre uno standard a riguardo ormai esistono diversi tipi di connettori. 
	\item \textbf{ChargeRequest}: Richiesta di ricarica. Viene istanziata quando un utente necessita di creare una prenotazione e al suo interno sono contenuti tutti i parametri necessari a descriverla. Fa parte del protocollo di richiesta di prenotazione discusso nella sezione ~\ref{sec:protocol}. Mentre un approfondimento sulla sua struttura è trattato nella sezione ~\ref{subsubsec:chargerequest}.
	\item \textbf{ChargeResponse}: È la risposta fornita dal servizio cittadino a seguito della richiesta di prenotazione. Al suo interno contiene un riferimento alla richiesta (ChargeRequest) da cui è stata generata e una lista di opzioni di ricarica che aderiscono alla richiesta dell'utente (ChargeOption).
	\item \textbf{ChargeOption}: Fa parte della risposta(ChargeResponse) che il servizio cittadino da all'utente in seguito a una richiesta di prenotazione (ChargeRequest). Contiene i parametri di ricarica come EVSE, orario e prezzo. 
	\item \textbf{Currency}: Rappresenta una valuta relativa a un prezzo. Alcune sue istanze sono state inserte direttamente nell'ontologia (\code{ioe:Euro}, \code{ioe:Dollar} ecc..);
	\item \textbf{Reservation}: Se il protocollo di richiesta di prenotazione va a buon fine verrà creata un istanza di questa classe che indica che l'EVSE a cui è associata è occupato per un determinato periodo di tempo. 
	\item \textbf{ReservationList}: Lista di prenotazioni associate ad un EVSE. Ogni EVSE può avere un unica lista di prenotazioni associata.
	\item \textbf{ReservationRetire}: Classe che denota la volontà dell'utente di ritirare una prenotazione.
	\item \textbf{Recharge}: Quando un utente, in seguito a una prenotazione, termina di ricaricarsi, inserita questa entità ad esso associato. Denota l'avvenuta ricarica è può essere utile per tener traccia dell'attività dell'utente nonché per fare statistiche.
	\item \textbf{UnityOfMeasure}: Rappresenta l'unità di misura per i dati del progetto. Deve esserne associata una ad ogni sottoclasse di \code{ioe:Data}. Attualmente sono hardcoded all'interno dell'ontologia (\code{ioe:Watt}, \code{ioe:Volt} ecc..) 	
	\item \textbf{Data}: Questa classe rappresenta il concetto di dato misurabile e ogni sua sottoclasse sarà caratterizzata da un valore e da un unità di misura.
\end{itemize}

\subsection{Sottoclassi di ioe:Data}\label{subsec:ioe-data}


\subsection{Modifiche apportate all'ontologia}

\subsubsection{ChargeRequest}\label{subsubsec:chargerequest}

\subsubsection{ChargeRequest}\label{subsubsec:chargeresponse}
	
	\clear
	  
	\chapter{Applicazione Mobile}

In questo capitolo verrà presentato un esempio di applicazione mobile in grado di connettersi al \emph{City Service} e di eseguire le operazione di prenotazione e ritiro delle ricariche. Il suo scopo è permettere all'utente di eseguire operazioni di interazioni con la smart-city al fine di ridurre le problematiche derivanti dall'utilizzo di veicoli elettrici. In particolare permette di eseguire richieste e cancellazioni di prenotazioni di ricariche attraverso i protocolli visti nella sezione ~\ref{sec:protocol}.

Inizialmente si limitava solo a queste operazioni ovvero prenotazione e cancellazioni di ricariche e inoltre funzionava solo in presenza del simulatore (~\ref{chap:sim}) in quanto prendeva il possesso di un veicolo gestito dal simulatore stesso. Il funzionamento è stato poi ampliato con la possibilità di connettersi tramite \emph{Bluetooth} a un veicolo reale, opportunità concessa dal \emph{Centro Ricerche Fiat} (\emph{CRF}), oppure di connettersi in assenza di veicoli.

A questo si è aggiunta la possibilità di analizzare il profilo altimetrico che separa il dispositivo mobile da un determinato EVSE con lo scopo di fare previsioni più accurate sui consumi necessari a raggiungerlo.

La piattaforma di sviluppo scelta è \emph{Android} vista la sua grandissima diffusione e versatilità.

\section{Architettura}

La piattaforma scelta per lo sviluppo è Android dalla versione \emph{4.0.3} in su. Questo perché vanta maggiori performance e un interfaccia utente più bella è più facile da programmare. La libreria di base per interfacciarsi con il SIB è quella esposta nella sezione ~\ref{subsec:ioe-lib}.

\subsection{Interazione con L'esterno}

L'interazione con il servizio cittadino avviene tramite scambio di messaggi con il \emph{City SIB}, mentre le informazioni relative al veicolo, in particolar modo se quest'ultimo è simulato, arrivano dal \emph{Dash SIB}.


\section{Modalità di esecuzione}

L'applicazione al fine di adattarsi ai diversi scenari possibili offre molteplici modalità di esecuzione, questo per adattarsi alle specifiche richieste per le dimostrazioni del Remote ??Monitoring¿¿. La scelta della modalità di esecuzione avviene nella schermata iniziale come si può vedere in figura ~\ref{fig:main-activity}

\subsection{Simulazione}

Questa modalità permette di prendere il controllo di un veicolo contenuto nel simulatore, il quale deve essere avviato con appositi parametri che causino la scrittura dei parametri relativi ai veicoli sul \emph{Dash SIB}. Questo implica che una volta che i abbiamo premuto sul pulsante \emph{Connect} (Fig. ~\ref{fig:main-activity}), dobbiamo scegliere l'utente Luciano Bononi in quanto è l'utente di default usato dalle macchine del simulatore (Fig. ~\ref{fig:select-user}). Una volta selezionato l'utente ci troveremo nel menu principale con solo due opzioni (Fig. ~\ref{fig:main-menu}). Questo perché le altre opzioni sono subordinate alla scelta di un veicolo. Una volta selezionato il veicolo (Fig. ~\ref{fig:select-veh}) vedremo esattamente i parametri che esso ha nel simulatore (posizione, carica, potenza ecc..). La posizione segnata nella mappa dell'app sarà la stessa segnata su SUMO. L'interazione tra il veicolo 

\begin{figure}[H]
        \centering
        \begin{subfigure}[H]{0.3\textwidth}
                \adjincludegraphics[width=\textwidth,trim={0 {0.5\height} 0 0},clip]{assets/mobile-app-select-user.png}
                \caption{Selezione Utente.}
                \label{fig:select-user}
        \end{subfigure}%
        ~ %add desired spacing between images, e. g. ~, \quad, \qquad etc.
          %(or a blank line to force the subfigure onto a new line)
        \begin{subfigure}[H]{0.3\textwidth}
                \adjincludegraphics[width=\textwidth,trim={0 {0.5\height} 0 0},clip]{assets/mobile-app-main-menu.png}
                \caption{Menu principale senza nessun veicolo selezionato}
                \label{fig:main-menu}
        \end{subfigure}
        \begin{subfigure}[H]{0.3\textwidth}
                \adjincludegraphics[width=\textwidth,trim={0 {0.5\height} 0 0},clip]{assets/mobile-app-select-veh.png}
                \caption{Selezione veicolo}
                \label{fig:select-veh}
        \end{subfigure}
        \caption{Schermate Applicazione Mobile}
\end{figure}

\subsection{Con Blue Me}

\subsection{Senza Blue Me}


\begin{figure}
	\centering
	\includegraphics[width=0.3\textwidth]{assets/mobile-app-main.png}
	\caption{Schermata Principale con i parametri di connessione ai SIB e la scelta di modalità di esecuzione.}
	\label{fig:main-activity}
\end{figure}


\section{Richiesta di prenotazione}

\section{Activities}

	
	\clear
	  
	\chapter{Piattaforma di Simulazione}\label{chap:sim}

La piattaforma di simulazione è uno strumento di fondamentale importanza al fine di validare l'infrastruttura software proposta. Risulta inoltre essere un valido strumento per valutare l'impatto dell'introduzione della mobilità elettrica veicolare all'interno di un determinato contesto, grazie ad esso si può prevedere quanti veicoli sarà in grado di supportare la grid, quante colonnine saranno necessarie e che potenza dovranno avere. Risulta quindi uno strumento fondamentale sia sotto il punto di vista dell'amministrazione pubblica/cittadina, che può prevedere un piano urbanistico sostenibile, sia dal punto di vista dei gestori della rete elettrica, i quali potranno valutare la richiesta energetica di tale scenario ed eventualmente prevedere investimenti in quella direzione.

\section{Architettura}

Al fine di poter simulare gli innumerevoli aspetti legati all'Electrical Mobility sono stati usati diversi simulatori/tecnologie in simbiosi. In questa sezione verranno introdotte brevemente al fine di introdurre al resto della trattazione.

\subsection{SUMO}\label{sebsec:sumo}

SUMO (Simulator of Urban Mobility) è un simulatore Open Source e multi-piattaforma di traffico urbano progettato per simulare reti stradali di grandi dimensioni. Sviluppato in C++ è supportato principalmente dall'Institute of Transportation Systems at the German Aerospace Center. La simulazione è di tipo microscopico ovvero ogni veicolo è modellato in modo esplicito, ha un proprio itinerario e si muove individualmente attraverso la rete. Ogni aspetto relativo alla simulazione viene configurato attraverso file XML i quali descrivono la rete stradale, i parametri ed i percorsi di ogni singolo veicolo, ed eventualmente altri aspetti legati alla simulazione come i flussi di traffico oppure la descrizione degli edifici. 

SUMO permette di avviare la simulazione in due modalità:

\begin{itemize}
	\item \textbf{Visuale}: La modalità visuale permette di avere un riscontro visuale l'andamento della simulazione tramite un interfaccia che mostra la mappa della rete/città con vista dall'alto. Vengono mostrati tutti i veicoli ed è possibile accedere a tutti i parametri della simulazione. Vengono mostrati inoltre i semafori agli incroci, la segnaletica delle strade e, nel caso siano stati caricati, gli edifici della città. Tutto questo ovviamente impatta notevolmente sulle performance ma, al di la del gradevole effetto visivo, è utile per vedere come evolve la simulazione. Nel nostro caso, ad esempio, è servito per assicurarsi che i veicoli si fermassero alle colonnine, oppure per valutare la quantità di traffico generata in seguito all'inserimento di un determinato numero di veicoli. Molto utile è stato anche in fase di Demo per mostrare il funzionamento del nostro simulatore.
	\item \textbf{Testuale}: Con la modalità testuale vengono stampati nel terminale i messaggi di Warning ed Error nel terminale e se richiesto anche qualche messaggio di debug in più che indica gli step di avanzamento della simulazione. Dopo aver constatato che la simulazione si comporta come ci si aspetta tramite la modalità visuale si passa allora a questa modalità che ha performance assai maggiori. È quindi particolarmente indicata per le simulazioni di lunga durata.
\end{itemize}

\subsubsection{Tools}\label{sumo-tools}

I file XML che descrivono le simulazioni possono diventare molto complessi qualora si decida di simulare scenari realistici (come Bologna). SUMO mette a disposizione innumerevoli tool automatici per la generazione dei file di configurazione.  In questa tesi prenderemo in esame solo quelli che ci sono stati utili:

\begin{itemize}
 	\item \textbf{netconvert}: Genera file con estensione .net.xml della dove viene mappata la rete stradale. La generazione avviene in modo pseudo-casuale, tramite la definizione dei nodi e degli archi che definiscono il "grafo" della rete stradale oppure, come nel nostro caso, attraverso la conversione da formati esterni(OpenStreetMap, VISUM, VISSIM, OpenDRIVE, MATsim ecc..)
 	\item \textbf{polyconvert}: Genera file con estensione .poly.xml dove sono contenute le informazioni relative agli edifici, zone di verde, fiumi laghi ecc.. Anch'esse vengono importate dai file delle mappe in altri formati.
 	 \item \textbf{duarouter}: Genera file con estensione .rou.xml che descrivono per ogni veicolo il suo percorso, compresi tutti i suoi step intermedi. La generazione dei percorsi avviene applicando un algoritmo di cammino su grafi a scelta tra Dijikstra o A*. I punti di partenza e arrivo vengono generati casualmente da uno script in python messo a disposizione tra i tool di sumo (randomTrips.py).
\end{itemize}

\subsubsection{TRaCI}

TraCI (Traffic Controller Interface) è un modulo messo a disposizione da SUMO che permette di interagire con la simulazione in tempo reale tramite un protocollo Client/Server basato su TCP/IP. All'avvio della simulazione SUMO si mette in ascolto su una porta in attesa di messaggi, qualunque linguaggio che supporti il protocollo TCP/IP può dunque modificare lo stato della simulazione oppure ricevere notifiche sul cambiamento di variabili alle quali ci si può sottoscrivere. È proprio TraCI che farà da ponte tra SUMO e l'altro simulatore usato all'interno della nostra piattaforma.

\subsection{OMNeT++}
OMNeT++ è un ambiente OpenSource di simulazione a eventi discreti. È principalmente usato per la simulazione di reti di comunicazione, ma grazie alla sua architettura modulare ed estremamente flessibile è possibile utilizzarlo negli ambiti più disparati come la simulazione di sistemi informatici complessi, architetture hardware o, come nel nostro caso, per supporto alla simulazione veicolare.

Le simulazioni vengono modellate tramite l'impiego di componenti riusabili chiamati \emph{moduli} i quali possono essere combinati tra loro come dei blocchi LEGO.

I moduli possono essere connessi tra di loro attraverso i \emph{gates} e combinati insieme per formare dei moduli composti (compound modules). La comunicazione tra moduli normalmente avviene tramite message passing e i messaggi possono contenere strutture dati arbitrarie (a parte informazioni predefinite tipo i timestamp). Questi messaggi possono viaggiare attraverso percorsi predefiniti dai gates e dalle connections oppure essere inviati direttamente alla loro destinazione, quest'ultima scelta è molto utile nel caso delle comunicazioni wireless. 

I moduli, i relativi parametri e i collegamenti fra loro, vengono definiti tramite un linguaggio di alto livello (NED) in appositi file con estensione .ned, mentre la logica viene implementata in una corrispondente classe C++.

OMNeT++ viene distribuito con un IDE basato su Eclipse grazie al quale possono essere eseguite molte operazioni in modo visuale, come ad esempio la creazione e aggregazione di moduli.

Anche OMNeT++ mette a disposizione due modalità di esecuzione della simulazione una visuale (\emph{Tkenv}) e una testuale (\emph{Cmdenv}). La modalità visuale permette di vedere i moduli con i relativi messaggi che vengono scambiati, viene usata in fase di debug o in fase di Demo. La modalità testuale, ovviamente più performante e adatta alle simulazioni batch, mostra solo i messaggi di debug della simulazione insieme allo standard output dei moduli. Per i nostri scopi abbiamo usato solo la modalità testuale.

Un grande punto di forza di OMNeT++ sono gli strumenti messi a disposizione per l'analisi dei dati generati dalle simulazioni, che permettono di applicare, in tempo reale, trasformazioni e aggregazioni tra i set di dati e, in fine, visualizzare i risultati con varie tipologie di grafici: a barre, a linee, istogrammi e molti altri.

Ci sono due tipi di dato che si possono registrare in OMNeT++ i vettori e gli scalari, ne caso dei vettori si hanno i dati sul piano cartesiano, con il tempo come ascissa e il dato come ordinata, mentre nel caso degli scalari viene registrato un solamente un dato.

\subsection{Veins}\label{subsec:veins}

Veins è un framework OpenSource per la simulazione di reti veicolari IVC (Inter-Vehicular Communication). Utilizza OMNeT++ e SUMO in simbiosi. Si appoggia su MiXiM, un framework per OMNeT++, che implementa modelli per reti wireless fisse e mobili (reti di sensori wireless, reti ad hoc, reti veicolari ecc.). La comunicazione con SUMO avviene tramite TRaCI. Ogni volta che nella simulazione in SUMO viene aggiunto un veicolo Veins crea dinamicamente un corrispondente modulo OMNeT++ che permette di controllarlo sotto ogni aspetto (percorso, colore, velocità, accelerazione, parcheggio ecc..).

Il simulatore consiste in un modulo di Veins, il quale è stato opportunamente modificato al fine di avere un ambiente che contiene solo i componenti strettamente necessari allo scopo in quanto le performance sono determinanti al fine di poter avere dei risultati in tempi utili. Infatti sono stati rimossi da Veins i moduli necessari alla comunicazione wireless (nic80211 e ARP), il modulo per la gestione degli ostacoli (obstacles) che era utilizzato per la gestione dello shadowing delle reti wireless

\subsubsection{Il funzionamento di Veins}\label{subsubsec:veins-func}

Veins è il ponte tra OMNeT++ e SUMO e la comunicazione tra i due avviene tramite TraCI. In realtà "in mezzo" ai due simulatori si trova uno script python, \emph{sumo-launchd.py}, che sta in ascolto sulla prima porta libera che trova, in attesa che venga avviato Veins. Quando Veins viene avviato si connette a questo script il quale lancia SUMO, a questo punto inizia la sincronizzazione tra i due simulatori che avviene tramite "staffetta" come mostrato in Fig. ~\ref{fig:veins-state-machine}. Per garantire l'esecuzione sincrona a intervalli definiti Veins inserisce in un buffer tutti i comandi da inviare a SUMO (Fig.  ~\ref{fig:veins-sequence-diagram}). Ad ogni passo temporale, i comandi contenuti nel buffer vengono inviati. Ciò innesca l'avanzamento del corrispondente passo temporale nella simulazione del traffico stradale. Al termine dello step temporale di simulazione del traffico stradale, SUMO invia una serie di comandi con lo stato e la posizione di tutti i veicoli istanziati in risposta a Veins. Dopo l'elaborazione di tutti i comandi ricevuti Veins aggiunge i corrispettivi nodi per ogni nuovo veicolo introdotto nella simulazione e rimuove invece i nodi relativi ai veicoli che sono giunti a destinazione. A questo punto la simulazione può avanzare al prossimo step temporale.

\begin{figure}[H]
        \centering
        \begin{subfigure}[H]{0.5\textwidth}
                \includegraphics[width=\textwidth]{assets/veins-state-machine.jpg}
                \caption{Panoramica dei due simulatori abbinati. Macchina a stati di SUMO e i moduli di Veins.}
                \label{fig:veins-state-machine}
        \end{subfigure}%
        ~ %add desired spacing between images, e. g. ~, \quad, \qquad etc.
          %(or a blank line to force the subfigure onto a new line)
        \begin{subfigure}[H]{0.5\textwidth}
                \includegraphics[width=\textwidth]{assets/veins-sequence-diagram.jpg}
                \caption{ Diagramma di sequenza dei messaggi scambiati tra SUMO e Veins. L'esecuzione dei comandi è ritardata fino al successivo passo temporale in SUMO.}
                \label{fig:veins-sequence-diagram}
        \end{subfigure}
        \caption{Architettura Veins}
\end{figure}


\section{Modellazione della Simulazione}

L'intera simulazione viene incapsulata all'interno di una Network. Il Network è lo scenario da simulare, all'interno di esso si definiscono i moduli che compongono la simulazione. Come mostrato in Fig \ref{fig:module-scenario} sono sostanzialmente 4 i moduli che compongono la nostra simulazione:

\begin{itemize}
	\item \textbf{world}: È un modulo di tipo \code{BaseWorldUtility}, fornito da MiXiM, che rappresenta l'area che circoscrive lo scenario simulato. Come configurazione richiede di definire la grandezza dello scenario in metri. La mappa usata in SUMO non può essere più grande delle dimensioni definite in questo modulo.
	\item \textbf{manager}: È un modulo di tipo \code{TraCIScenarioManagerLaunchd}, fornito da Veins, che mette in comunicazione OMNeT++ con SUMO. Tutte i messaggi inviati a TraCI passano da questo modulo (l'implementazione vera e propria della comunicazione con TraCI avviene nel modulo padre \code{TraCIScenarioManager}). Questo modulo è fondamentale in quanto è quello che crea un modulo OMNeT++ per ogni veicolo di SUMO.
	\item \textbf{cityService}: Rappresenta la grid, infatti contiene i GCP e gli EVSE. Ricopre anche la funzione di raccoglitore statistiche globali sulle colonnine e i veicoli (Sez: \ref{sec:module-city}).
	\item \textbf{connectionManager}: Questo modulo si occupa della comunicazione tra moduli ma è inutilizzato. Non l'ho rimosso per questioni di compatibilità con Veins.
\end{itemize}

Ogni volta che SUMO crea un veicolo Veins si occupa di creare il corrispondente modulo in OMNeT++, il modulo in questione è Car (Fig. \ref{fig:module-car}). Car in realtà è un modulo composto ovvero un contenitore, privo di implementazione, che contiene altri moduli. Di default, conterrebbe tutti i componenti relativi alle comunicazioni wireless in quanto sarebbe il target di Veins, ma io li ho rimossi in quanto non utili, per ora, nel nostro scenario.

Al fine di rendere la simulazione il più possibile attinente alla realtà risulta necessario implementare i modelli di carica e scarica dei veicoli e i modelli comportamentali degli utenti nonché l'implementazione dei moduli di comunicazione con il servizio cittadino. Per svolgere questi compiti è stato necessario arricchire la definizione di Car con 3 nuovi moduli:

\begin{itemize}
	\item \textbf{TraciMobility}: È un modulo fornito da Veins che mette in comunicazione il veicolo con OMNeT++ tramite TRacI.
	\item \textbf{CarLogic}: È il modulo principale, in esso è implementata la logica del veicolo. 
	\item \textbf{Battery}: Implementa il modello di carica e scarica del veicolo.
	\item \textbf{DriverBehaviour}: In questo modulo sono implementati i comportamenti che l'utente assume dinanzi a determinate scelte.  
\end{itemize}

Ogni modulo contiene dei parametri che permettono di cambiarne il comportamento, grazie a OMNeT++ diventa semplice lanciare molteplici simulazioni con diversi set di dati. Particolarmente interessante è la possibilità di eseguire la simulazione senza prenotazione per poter confrontare le differenze di occupazione delle colonnine rispetto allo scenario con prenotazione.

\begin{figure}[H]
        \centering
		\begin{subfigure}[H]{0.45\textwidth}
                \includegraphics[width=\textwidth]{assets/module-scenario.jpg}
                \caption{Modulo scenario}
                \label{fig:module-car}
        \end{subfigure}
        \qquad
        \begin{subfigure}[H]{0.45\textwidth}
                \includegraphics[width=\textwidth]{assets/module-car.png}
                \caption{Modulo Car}
                \label{fig:module-scenario}
        \end{subfigure}%
        \caption{Moduli OMNeT++}
\end{figure}

\subsection{Ciclo di Vita dei moduli}\label{subsec:lifecycle}

Essendo OMNeT++ un simulatore a eventi discreti i moduli, in esso implementati, non fanno nulla finché non viene schedulato un evento. Gli eventi vengono schedulati attraverso messaggi inviati dai moduli stessi. Sostanzialmente si tratta di decidere quando, a chi e cosa mandare.
Il quando è il tempo di simulazione e quindi deve essere nel futuro o al massimo nel momento attuale di simulazione, a chi è il modulo destinatario e cosa è il messaggio da mandare. OMNeT++ fornisce un un messaggio di base, \code{cSimpleMessage}, ma, come vedremo più avanti, è possibile definirsi messaggi propri più complessi.

All'interno di questa simulazione non avviene molto scambio di messaggi tra i diversi moduli, ragion per cui risulta necessario auto-inviarsi i messaggi al fine di mantenere "vivo" il modulo.

Quando Veins crea un modulo il corrispondente veicolo in SUMO viene aggiornato ogni 0.1 secondi, che è lo step temporale di default. In OMNeT++ invece siamo noi a decidere ogni quanto aggiornare un modulo attraverso il meccanismo degli auto-messaggi. Schedulando i messaggi in modo intelligente si può guadagnare in performance in quanto ci si può inviare il messaggio solo quando è realmente necessario.

Quando un modulo Car viene creato viene chiamata la funzione \code{initialize()}, della quale il programmatore può eseguire l'override, nella quale dopo le opportune inizializzazioni, è necessario auto-schedularsi un messaggio. I messaggi vengono ricevuti dalla funzione \code{handleMessage()} alla quale viene passato un riferimento al messaggio stesso. Da dentro questa funzione si può implementare la logica del modulo. Il modulo continua a vivere finché, il corrispondente veicolo in SUMO, non giunge a destinazione momento in cui Veins, attraverso la classe \code{TraCIScenarioManager}, si accorge che il veicolo non è più nella simulazione e quindi lo elimina anche da OMNeT++ il che causa una chiamata alla funzione di terminazione \code{finish()}.

\subsection{CityService}\label{sec:module-city}

Il modulo CityService simula la grid, al suo avvio carica le informazioni relative ai GCP presenti nello scenario simulato da un file XML. Il file utilizzato è lo stesso del servizio cittadino (Sez. ~\ref{subsubsec:city-init}). Le colonnine caricate vengono trasformate in oggetti C++ al fine di poter essere usate dai veicoli virtuali. 

Questo modulo, attraverso il settaggio di parametri esterni, provvede ad abilitare/disabilitare le prenotazioni, e inoltre decide il tasso di penetrazione di veicoli elettrici nello scenario.

Quelli mostrati di seguito sono i parametri del modulo:

\begin{itemize}
	\item \code{gcpList}: Percorso del file XML che contiene la definizione dei GCP. Deve essere lo stesso utilizzato dal servizio cittadino reale.
	\item \code{electricalVehicleFreq}: Questo parametro stabilisce la frequenza con cui viene immesso un veicolo elettrico nella simulazione. È un numero che può variare da 0 a 100 e, più precisamente, indica con quale probabilità il veicolo inserito nella simulazione è elettrico. È necessario ricordare che il numero e la frequenza con cui i veicoli vengono immessi nella simulazione è determinato da SUMO attraverso i suoi file di configurazione. 
	\item \code{maxElectricalVeh}: Impone un limite superiore al numero di veicoli che possono essere presenti nella simulazione in un determinato istante. Questo significa che se è stato raggiunto il massimo numero di veicoli, e uno di questi lascia la simulazione per un qualunque motivo, allora verrà rimpiazzato da uno nuovo (sempre ammesso che SUMO generi altri veicoli e che la statistica sia favorevole). Se viene impostato a -1 allora non ci sarà nessun limite al numero di veicoli.
	\item \code{reservationEnabled}: Abilita/Disabilita il protocollo di prenotazione per i veicoli. l fine di ottimizzare le performance lo scambio di dati con il SIB, necessario per le prenotazioni, viene completamente disabilitato quando quest'ultime non sono attive.
\end{itemize}

Altra importante funzione svolta da questo modulo è la raccolta di statistiche globali sui veicoli elettrici. I dati raccolti sono tutti in formato vettoriale:

\begin{itemize}
	\item \textbf{chargingVehicles}: In questo vettore viene salvato lo stato di occupazione degli EVSE della città, ogni volta che un veicolo si va a ricaricare aggiunge un unità al vettore e quando finisce la ricarica l'unità viene rimossa. Questo comporta che come ordinata avremo al massimo il numero totale di EVSE presenti nello scenario.
	\item \textbf{electricalVehicles}: Numero di veicoli elettrici presenti nella simulazione, semplicemente ogni volta che viene aggiunto un veicolo elettrico alla simulazione viene incrementata di un unità il vettore.
	\item \textbf{vaporizedVehicles}: I veicoli vaporizzati sono quei veicoli che hanno terminato la batteria e quindi vengono letteralmente vaporizzati. Il termine vaporizzati deriva dall'analogo comando di TRacI che permette di rimuovere un veicolo dalla simulazione. 
	\item \textbf{leavingVehicles}: Questo veicolo tiene traccia dei veicoli che riescono a lasciare normalmente la simulazione ovvero arrivano a destinazione, evento che ne causa la rimozione da parte di SUMO. In realtà uno degli obbiettivi raggiunti è stato proprio quello di dirottare i veicoli che arrivano a destinazione verso una strada casuale. Non è sempre possibile intercettare l'arrivo del veicolo in una determinata strada, questo comporta che qualcuno di essi "sfugga" e venga rimosso dalla simulazione.
\end{itemize}


\subsection{CarLogic}

Il comportamento del veicolo è definito dal modulo CarLogic. Questo modulo implementa tutta la logica relativa alla guida del veicolo. In esso sono contenute le informazioni che ne descrivono la tipologia, l'appartenenza, e alcuni comportamenti di base. I comportamenti più complessi sono delegati al modulo DriverBehviour.

\subsubsection{I parametri}

\begin{itemize}
	\item \code{userName}: Nome dell'utente che possiede il veicolo
	\item \code{userId}: Identificativo dell'utente che possiede il veicolo
	\item \code{manufacturer}: Casa produttrice del veicolo
	\item \code{model}: modello del veicolo
	\item \code{cRoll}: coefficiente resistenza attrito gomme su asfalto
	\item \code{cDrag}: coefficiente di forma che indica la levigatezza del veicolo
	\item \code{rhoAir}: coefficiente di resistenza del dll'aria
	\item \code{across}: Sezione frontale del veicolo
	\item \code{weight}: peso del veicolo
	\item \code{threshold}: soglia sotto la quale il veicolo si considera scarico e quindi diviene necessario fare una ricarica. Varia da 0 a 1.
	\item \textbf{minRequestedEnergyKwh}: Quantità minima di energia richiedibile in una richiesta di prenotazione. Questo valore serve nei casi in cui una richiesta non venga accettata dal \emph{City Service}, in tal caso viene diminuita gradualmente la quantità di energia richiesta fino ad arrivare a questa soglia.
	\item \textbf{writeCarStatusOnSib}: Dice se scrivere le informazioni di stato dei veicoli sul \emph{Dash SIB}. Questo permette di vedere lo stato dei veicoli dall'applicazione mobile. Essendo la scrittura sul SIB un operazione abbastanza onerosa è meglio tenere disattivata questa opzione a meno che non si sia in fase di demo.
\end{itemize}


\subsubsection{Inizializzazione}

In fase di inizializzazione la prima cosa che viene fatta è prendere un riferimento a tutti i moduli necessari, tra i quali il CityService, che  deciderà se il veicolo sarà elettrico o meno. Nel caso in cui sia elettrico allora si procede a reperire tutti i parametri e, se è abilitata la prenotazione, vengono scritte le informazioni del veicolo sul \emph{Dash SIB}. Viene inoltre mandato un comando a SUMO colora il veicolo di verde, funzionalità molto utile sia in fase di debug che in fase di demo.

Nel caso in cui il veicolo non sia elettrico allora vengono eliminati i relativi moduli da OMNeT++ ma il veicolo rimane in SUMO. Questa funzionalità non era prevista da Veins e quindi è stato necessario modificarlo opportunamente per introdurla. L'eliminazione avviene indirettamente, dal momento che un modulo di Veins non può eliminare se stesso, mandando un messaggio al CityService con la richiesta di eliminazione.

Un problema di SUMO, almeno per quelli che sono i nostri obbiettivi, è che un veicolo una volta giunto alla sua destinazione viene eliminato dalla simulazione. Questo comportamento influisce negativamente sulla simulazione in quanto a noi interessa simulare un periodo di vita dei veicoli lungo abbastanza da poterne studiare diversi cicli di carica e scarica. Per sopperire a questo problema vengono ottenuti, tramite TraCI, gli ID di tutte le strade che compongono il percorso del veicolo e viene preso l'ultimo in modo da poter intercettare l'arrivo del a quest'ultimo e quindi dirottarlo verso un'altra destinazione casuale. Le altre destinazioni vengono scelte da una lista che viene riempita con gli ID elle strade di destinazione, prese dai veicoli stessi, più lunghe di 50 metri. Il controllo sulla lunghezza serve in quanto è più probabile intercettare il momento in cui il veicolo arriva  a destinazione.

A questo punto come ultima operazione viene istanziato un messaggio di tipo \code{CarMessage}, appositamente creato per mantenere lo stato del veicolo attraverso le varie transazioni di stato. I campi del messaggio vengono inizializzati e il messaggio viene schedulato al tempo attuale di simulazione. Come si vede nel List. \ref{lst:omnet-msg} il messaggio viene creato, viene impostato lo stato del veicolo (Sez. \ref{subsubsec:veh-state}), e infine viene schedulato tramite la funzione \code{scheduleAt()} al tempo attuale di simulazione che viene fornito dalla funzione \code{simTime()}. Per schedulare il messaggio dopo 25 secondi sarebbe stato necessario usare \code{simTime() + 25}.

\begin{cpp}[caption={Autoschedulazione Messaggio},label={lst:omnet-msg}]
carMessage = new CarMessage("CarMessage");
carMessage->setCarState(CarState::DRIVING);
[...]
scheduleAt(simTime(), carMessage);
\end{cpp}


\subsubsection{Gli stati del veicolo}\label{subsubsec:veh-state}

Lo stato del veicolo è definito da un automa a stati finiti come mostrato in Fig. \ref{fig:car-fsmd}. Le transazioni tra gli stati avvengono tramite scambio di messaggi nei quali è definito lo stato successivo. I messaggi arrivano alla funzione \code{handleMessage} la quale controlla se il messaggio arriva dall'esterno oppure se è auto-inviato (\code{msg->isSelfMessage()}). In quest'ultimo caso allora il messaggio viene inoltrato a una funzione chiamata \code{handleSelfMessage()}. All'interno di quest'ultima funzione avviene la scelta di quale hanlder eseguire in base allo stato definito nel messaggio (Lst. \ref{lst:self-msg}).

\begin{cpp}[caption={Funzione di scelta dello stato}, label={lst:self-msg}]
void CarLogic::handleSelfMessage(cMessage *msg) {
	CarMessage* carMsg = check_and_cast<CarMessage *>(msg);
	
	switch (carMsg->getCarState()) {
		case CarState::DRIVING:
			handleDriving(carMsg);
			break;
		[...]
		case CarState::CHARGING:
			handleCharging(carMsg);
			break;
		default:
			error("Unknown Car State!");
			break;
	}	
}
\end{cpp}

Ogni stato del veicolo ha una sua funzione "handler" che ne determina il comportamento. Alla funzione viene passato un riferimento al messaggio che contiene informazioni sullo stato del veicolo. L'handler prima di finire rischedula il messaggio con un nuovo stato, o con lo stesso in alcuni casi. La Fig. \ref{fig:car-fsmd} mostra l'automa a stati finiti che descrive il veicolo. 

I possibili stati del veicolo sono definiti nell'enumerazione \code{CarState}. Sarà quindi presente, ad esempio, la funzione \code{handleDriving} associata allo stato \code{CarState::DRIVING}, e così via per tutti gli stati del veicolo.

Dal momento che l'implementazione C delle KPI, le librerie che si interfacciano con il SIB tramite il protocollo SSAP (Sez. \ref{sec:smart-m3}), non supporta le sottoscrizioni, che sono alla base della comunicazione con il CS, risulta necessario fare il ``polling'' a intervalli regolare per ricavare le risposte. A questo compito sono stati riservati due stati del veicolo \code{WAITING_RESPONSE} e \code{WAITING_CONFIRM}. 

\begin{description}
	\item \label{state:driving} \code{DRIVING}: Quando il veicolo si trova in questo stato significa che si sta dirigendo verso la sua destinazione. Ogni volta che viene eseguito fa un controllo sullo stato di carica della batteria e se questa è inferiore alla soglia stabilita dal parametro \code{threshold} allora il veicolo si considera scarico e quindi riuslta necessario dirigesi a una colonnina. A questo punto si presentano due casistiche:
	\begin{itemize}
		\item{Con Prenotazione}: se la simulazione è stata eseguita con le prenotazioni attive allora il veicolo deve eseguire il protocollo di prenotazione. Vengono quindi create le triple necessarie a istanziare una richiesta di prenotazione e inserite nel SIB. Se la richiesta va a buon fine allora si imposta come stato successivo \code{WAITING_RESPONSE}, altrimenti viene rischedulato questo e reiterata la richiesta.
		\item{Senza Prenotazione}: Se la simulazione è stata eseguita senza prenotazione allora viene scelto casualmente un GCP tra i 3 più vicini e il veicolo dirige direttamente verso quello passando nello stato \code{GO_TO_RECHARGE}.
	\end{itemize}
	\item \code{WAITING_RESPONSE}: il veicolo interroga il SIB in cerca della risposta da parte del CS. La ricerca è eseguita tramite query SPARQL. Se fallisce, oppure restituisce un risultato vuoto (non sono disponibili opzioni di ricarica conformi alla richiesta), il veicolo torna in stato \code{DRIVING} dove la richiesta viene reiterata diminuendo l'energia richiesta e aumentando il lasso di tempo in cui si è disposti a caricarsi. Se la query restituisce una risposta allora vengono analizzate le opzioni di ricarica in essa contenute e, in base al comportamento definito in \code{DriverBehaviour}, ne viene scelta una. Lo stato successivo sarà  \code{WAITING_CONFIRM}
	\item \code{WAITING_CONFIRM}: il veicolo interroga il SIB in cerca della conferma da parte del CS. Analogamente a quanto succede nello stato precedente se la ricerca fallisce ho ha esito negativo si ritorna in stato \code{DRIVING}. In caso di successo è impostata come destinazione la strada del GCP dove avverrà la ricarica. Se il tempo che manca alla ricarica è maggiore di un quarto d'ora allora il veicolo passa in stato \code{PARKING} in attesa dell'orario della ricarica, questo comportamento serve ad evidenziare il fatto che un utente che dispone della prenotazione è meno ansioso rispetto a uno che invece deve girare tutti i GCP della città fino a che non trova un EVSE libero. Se il tempo che manca alla ricarica è minore di un quarto d'ora allora il veicolo si dirige verso al GCP passando in stato \code{GO_TO_RECHARGE}
	\item \code{PARKING}: il veicolo si parcheggia nella prima strada disponibile in attesa del tempo necessario alla ricarica. Lo stato successivo viene definito nello stato precedente a questo. Attualmente il solo stato che porta a \code{PARKING} è \code{WAITING_CONFIRM} e come stato successivo imposta \code{GO_TO_RECHARGE}.
	\item \code{GO_TO_RECHARGE}: il veicolo si dirige verso il GCP. Controlla periodicamente la strada in cui si trova e quando arriva in quella del GCP  si ferma. A questo punto si presentano due casistiche:
	\begin{itemize}
		\item{Con Prenotazione}: Se la colonnina è occupata allora il veicolo, che evidentemente è arrivato in anticipo, si mette in coda in attesa del suo turno passando nello stato \code{WAITING_FOR_EVSE} altrimenti inizia la ricarica e si passa in stato \code{CHARGING}.
		\item{Senza Prenotazione}: Si controlla se ci sono degli EVSE liberi e in tal caso inizia la ricarica passando in stato \code{CHARGING}. In assenza di EVSE allora si interroga il modulo \code{DriverBehaviour} che stabilisce se fermarsi ad aspettare il veicolo che si sta caricando attualmente passando nello stato \code{WAITING_FOR_EVSE}. In caso contrario si sceglie casualmente un GCP tra i tre più vicini che non siano già stati visitati e ci si dirige verso di esso passando nello stato \code{GO_TO_RECHARGE}. 
	\end{itemize}
	\item \code{WAITING_EVSE}: questo stato rappresenta l'attesa presso un EVSE. Non viene schedulato nessun messaggio poiché è compito del veicolo attualmente in carica ``risvegliare'' il veicolo in attesa. Lo stato successivo al ``risveglio'' è \code{CHARGING}
	\item \code{CHARGING}: Rappresenta la carica lo stato di carica di un veicolo. Si rimane in questo stato finché la carica non è completa oppure, nel caso di prenotazione attiva, fino al raggiungimento del tempo di fine ricarica.
\end{description}


Al termine del ciclo di vita del modulo si provvede all'eliminazione dei dati relativi al veicolo dal SIB.
	
\begin{figure}
	\centering
	\includegraphics[width=1.0\textwidth]{assets/car-fsmd.png}
	\caption{Automa a stati finiti che descrive il veicolo, tutti gli stati possono essere finali.}
	\label{fig:car-fsmd}
\end{figure}

\subsection{Battery}\label{sec:battery}

Il modulo \code{Battery} fornisce il modello di consumo energetico del veicolo. Non è esatto parlare di modello della batteria dal momento che le equazioni usate sono relative al lavoro e non strettamente legate all'elettrotecnica poiché troppo complesse da simulare con il poco tempo a disposizione. Il modello implementato rimane abbastanza fedele alla realtà in virtù del fatto che è stato validato con dati sperimentali raccolti durante la visita al CRF.

Determinante al fine di modellare e validare il consumo è stato il lavoro di Alfredo D'Elia prima e Marco di Nicola (\cite{dinicola2014}) dopo. Il modello attualmente tiene conto della pendenza della strada grazie alla quale si può calcolare l'energia richiesta in salita e quella guadagnata in discesa. A questo si aggiunge la ricarica dovuta alla frenata rigenerativa.

Il ciclo di vita del modulo batteria è determinato dall'auto-invio di messaggi la quale schedulazione è indipendente da quella del modulo \code{CarLogic}.

Questo modulo può assumere due stati: \code{CHARGING} e \code{DISCHARGING} definiti nell'enumerazione \code{BatteryState}. Le relative funzioni handler sono \code{handleDischarge} e \code{handleCharge}. Lo stato della batteria viene modificato da \code{CarLogic}. Quando si arriva a una colonnina si imposta lo stato \code{CHARGING}. Viene inoltre passato a \code{Battery} un riferimento all'EVSE corrispondete il quale fornisce le informazioni sulla potenza della colonnina e quindi del tempo necessario a caricarsi. Al termine della ricarica \code{CarLogic} imposta lo stato del modulo batteria in \code{DISCHARGING} e dirige il veicolo verso una destinazione casuale.

Il modulo \code{Battery} ha anche la funzione di eliminare il veicolo dalla simulazione quando si esaurisce la carica della batteria.

\subsubsection{I parametri}

Vengono mostrati i parametri che caratterizzano il modulo batteria. I valori di default sono relativi al Fiat Daily in quanto, avendo a disposizione i dati sperimentali, hanno permesso di validare al meglio i dati generati dalla simulazione.

\begin{itemize}
	\item \textbf{manufacturer}: costruttore della batteria, di default impostato a Simens in quanto partner del progetto IoE;
	\item \textbf{engineEfficense}: efficienza della gestione energetica del motore, varia tra 0 e 1.
	\item \textbf{socPerc} =  carica iniziale del veicolo, varia da 0 a 1, dove indica una quantità di carica pari alla capacità della batteria;
    \item \textbf{capacity} = capacità della batteria misurata in Kilowattora (kWh);
    \item \textbf{maxVoltage} = voltaggio massimo erogato misurato in Volt (V);
    \item \textbf{voltage} = voltaggio attualmente erogato  in Volt (V);
    \item \textbf{stateOfHealth} = stato di vita della batteria, varia tra 0 e 1. Attualmente non considerato nel modello;
    \item \textbf{maxCurrentIn} = massima intensità di corrente in entrata misurata in Ampere (A);
    \item \textbf{maxCurrentOut} = massima intensità di corrente in uscita misurata in Ampere (A);
    \item \textbf{nominalTempearture} = massima temperatura a cui può lavorare il motore misurata in Gradi Celsius (\textcelsius). Attualmente non considerata nel modello.
    \item \textbf{recordVectors} = Indica se registrare o meno le statistiche sulla batteria. Questi dati tendono a far crescere molto la dimensione dei file che li contengono;
    \item \textbf{regenerativeBrakingEfficiency} = efficienza della frenata rigenerativa;
\end{itemize}

\subsubsection{Consumo del veicolo}

Il modello di consumo è implementato nella funzione \code{handleDischarge()}. I dati necessari al calcolo sono: 

\begin{itemize}
	\item \textbf{Velocità attuale}: fornita da Veins che ne tiene bufferizzato il valore senza eseguire comandi tramite TraCI. A volte, quando il veicolo è fermo, Veins ricorda il valore della velocità precedente alla fermata provocando un consumo della batteria anche a veicolo parcheggiato. Risulta quindi necessario verificare manualmente che il veicolo non sia parcheggiato.
	\item \textbf{Accelerazione}: siccome non fornita attraverso TraCI è necessario calcolarla come differenza tra la velocità attuale e quella registrata nell'ultima chiamata a \code{handleDischarge()} diviso il tempo simulato trascorso.
	\item \textbf{Angolo di inclinazione stradale}: l'angolo di inclinazione stradale ($\alpha$) è fornito dalla funzione \code{getInclinationRad()} che lo ricava a partire da 2 punti nel piano tridimensionale ricavati tramite la funzione \code{getLatLonAlt()} di \code{CarLogic} attraverso una doppia chiamata a SUMO. La prima necessaria ad ottenere la nostra posizione sulla strada corrente, la seconda ad effettuare una conversione di quest'ultima in latitudine, longitudine, altitudine. Fortuna vuole che questa feature è stata recentemente introdotta in SUMO.
	\item \textbf{La quantità di carica}: dal momento che i calcoli restituiscono risultati in kWh è necessario convertire la quantità di carica espressa in percentuale in kWh con la formula: $soc_{Kwh} = soc_{\%} \cdot capacity$.
\end{itemize}

Le funzioni necessarie a calcolare distanza e inclinazione a partire da punti in 3 dimensioni sono state parzialmente ereditate dalla librerie UniboGeoTools e si trovano nel file \code{Util.cc}. Si noti che la distanza viene calcolata attraverso applicazione di proiezione ortogonale (\code{getEquirectangular ApproximationDistance()}) che non è la formula più precisa per la distanza tra due punti espressi in coordinate latitudine/longitudine. Ma viste le brevi distanze in gioco, quindi il ridotto errore che ne deriva, e la maggiore velocità di esecuzione, si è optato per il suo utilizzo al posto di formule più precise come la formula dell'emisenoverso.

Sono considerate tre forme diverse di dispendio energetico nello spostamento del veicolo:

\begin{itemize}
	\item \textbf{$L_m$}: lavoro compiuto dal motore per far muovere il veicolo.
	\item \textbf{$L_g$}: lavoro necessario a superare l'attrito delle gomme sull'asfalto.
	\item \textbf{$L_a$}: lavoro necessario a superare la resistenza dell'aria.
\end{itemize}

\noindent
Considerando:

\begin{center}
	$\boxed{m \cdot a = F - m \cdot g \cdot \sin{\alpha} }$
\end{center}

\noindent
La forza necessaria a spostare il veicolo tenendo conto della pendenza è:

\begin{center}
	$\boxed{F = m \cdot (a + g) \cdot \sin{\alpha} }$
\end{center}

\noindent
Ne segue che il lavoro necessario a spostare il veicolo è:

\begin{center}
	$\boxed{L_m = \frac{1}{engineEfficiency} \cdot \Delta t \cdot F \cdot \frac{V + V_{old}}{2}}$
\end{center}

\noindent
La massa del veicolo ($m$) è data dai parametri del modulo \code{CarLogic}, precedentemente introdotti. L'efficenza del motore è definita tra i parametri del modulo \code{Battery}.
\\
\noindent
Il lavoro $L_g$, necessario a vincere l'attrito delle gomme sull'asfalto, è dato da:

\begin{center}
$\boxed{L_g = \frac{1}{engineEfficiency} \cdot \Delta t \cdot (m \cdot g \cdot \cos{\alpha} \cdot cRoll) \cdot \frac{V + V_{old}}{2}}$
\end{center}

\noindent
$cRoll$, introdotto in \code{CarLogic}, è l'attrito delle gomme sull'asfalto.
\\
\noindent
Infine il lavoro necessario a vincere la resistenza dell'aria è:

\begin{center}
	$\boxed{L_a = \frac{1}{engineEfficiency} \cdot \Delta t \cdot \frac{(\rho_{air} \cdot cDrag \cdot across)}{2} \cdot 	(\frac{V + V_{old}}{2})^3}$
\end{center}

\noindent
Ne consegue che il consumo energetico totale, espresso in kWh, sia dato da:

\begin{center}
	$\boxed{energyConsumption = \frac{L_m + L_g + L_a}{3600000}}$
\end{center}

\noindent
che viene sottratto alla quantità di carica in kWh precedentemente calcolata che viene ritrasformata in percentuale.

\subsubsection{Frenata Rigenerativa}

Il freno rigenerativo è un particolare tipo di freno che recupera energia utile estraendola da una quota di quella che, normalmente, si dissipa in aria (calore) durante il rallentamento del veicolo (diminuzione di energia cinetica). Nel nostro sistema la frenata è determinata da accelerazione negativa. In questo caso $L_m$, che andrebbe a contribuire all'energia consumata dal veicolo, non viene considerato e al suo posto viene considerata l'energia cinetica prodotta nella decelerazione del veicolo in Joule (J):

\begin{center}
	$\boxed{E_{kin} = \frac{1}{2} \cdot m \cdot (V^2 - V_{old}^2) \cdot \eta_{rig}}$
\end{center}

$\eta_{rig}$ è la percentuale di energia recuperabile da quella prodotta indicata con il parametro di \code{Battery} \code{regenerativeBrakingEfficiency}. Questo dipende dell'efficienza del meccanismo di frenata/rigenerazione che ha un un limite teorico del 30\% (valore di default)  prodotto da recenti studi del IDSC (Institute for Dynamic Systems and Control) svizzero.

Determinante, al fine di raggiungere questo risultato, è stato il contributo di Marco Di Nicola.

\subsubsection{Ricarica del veicolo}

Il modello di ricarica è implementato nella funzione \code{handleCharge()}. Per determinare la carica è necessario avere le informazioni sull'EVSE al quale si è attaccati. Un istanza di un oggetto che rappresenta l'EVSE viene passato al modulo \code{Battery} dal modulo \code{CarLogic}.

Viene presa la potenza della colonnina espressa in kW  ($power_{kW}$) e a ogni iterazione il guadagno energetico viene calcolato nel seguente modo:

\begin{center}
	$\boxed{soc_{kWh} =  soc_{kWh} + (power_{Kw} \cdot \Delta t) \cdot 3600}$
\end{center}

\noindent
Quando la carica raggiunge la capacità della batteria il processo viene interrotto e il veicolo lascia la colonnina.


\subsection{DriverBeahviour}

Il modulo \code{DriverBeahviour} si occupa di modellare alcuni comportamenti dell'autista del veicolo. Attualmente non è particolarmente evoluto in quanto essendo stato introdotto dopo \code{CarLogic} parta della logica è implementata in quest'ultimo.

\subsubsection{I parametri}



\section{Implementazione}

????
Esisteva già una versione del simulatore ma era a puro titolo dimostrativo e di demo e soffriva del fatto che era stato sviluppato da diverse persone (me compreso) in tempi molto brevi e con deadline che corrispondendo a demo internazionali si era costretti a rispettare. Questo ha portato ad avere un codice farraginoso e pieno di memory leak. Basti pensare che la prima versione del simulatore allocava RAM esponenzialmente e già dopo 2000 secondi si poteva arrivare ad avere un occupazione di 4GB.

La prima cosa che ho fatto, quando ho capito che la situazione stava diventando ingestibile è stato profilare e rifattorizzare il codice. La profilazione è avvenuta tramite il tool Valgrind grazie al quale sono riuscito ad ottenere un cosumo di memoria lineare, infatti, dove prima venivano occupati 4GB, sono riuscito a raggiungere il traguardo dei 100MB. La rifattorizzazione del codice invece è stata più complessa in quanto diverse mani hanno messo mano con diversi stili di programmazione. 

\subsection{Logging}

\subsection{SibController}

\subsection{GcpController}

\subsubsection{GCP e EVSE}

\subsection{Utility}

\section{L'ambiente di simulazione}

In questa sezione verranno descritti in dettaglio i componenti necessari a creare un ambiente di simulazione funzionante.

La logica del simulatore è implementata attraverso moduli di OMNeT++. Grazie ad essi sono implementati, i modelli di consumo dei veicoli elettrici, i comportamenti degli degli autisti e la rete di distribuzione elettrica cittadina. L'unico aspetto non implementato è la guida dei veicoli in quanto è gestita da SUMO.

I file di configurazione di SUMO sono generati da script che in base ai parametri specificati possono variare l'intensità del traffico.


\subsection{Generazione file di Configurazione}

Dopo aver scaricato compilato ed installato tutti i componenti è necessario generare i file di configurazione riguardanti lo scenario che si vuole simulare. 

\subsection{Download Scenario}

A questo punto è necessario scegliere quale scenario si vuole simulare. Lo scenario di Bologna è già disponibile nella cartella \code{simulator/veins-2.1/examples/veins/bologna} siccome è quello di nostro interesse.

Nel caso in cui si sia interessati ad uno scenario diverso da quello di Bologna il modo più semplice per ottenere la mappa desiderata è andare all'indirizzo \url{http://www.openstreetmap.org/export} e scaricarsi l'area interessata. La dimensione delle mappe scaricabili è limitata onde evitare la saturazione della banda del server. Per sopperire a questa mancanza SUMO mette a disposizione un tool situato in \code{<SUMO_HOME>/tools/import/osm/osmGet.py} che permette di scaricare mappe di dimensione arbitraria. Per l'utilizzo di questo tool rimando alla documentazione dello script oppure alla pagine ufficiale:

\url{http://sumo-sim.org/userdoc/Networks/Import/OpenStreetMapDownload.html}.

\subsubsection{Profilo Altimetrico}\label{profilo-altimetrico}

Da notare che le mappe di Open Street Map non contengono le informazioni relative al profilo altimetrico. È quindi necessario "arricchire" la mappa scaricata con tali informazioni. Il programma utilizzato a questo scopo è Osmosis, presente nella cartella \code{osmosis} del progetto. In particolare ho usato \code{osmosis-srtm-plugin_1.1.0} che permette, attraverso l'interrogazione di file SRTM (scaricabili da \url{http://dds.cr.usgs.gov/srtm/version2_1/SRTM3/}, di inserire i dati del profilo altimetrico nelle mappe di Open Street Map. 

Di seguito viene mostrato l'utilizzo del Osmosis e del relativo plugin considerando \code{\$SRTM_HOME} la cartelle che contiene i file SRTM e \code{\$CITY_NAME} il nome della città. Quindi avendo, ad esempio, \code{bologna.osm}, ovvero la mappa della città di Bologna senza dati riguardanti il profilo altimetrico, in output avremo \code{bologna_srtm.osm}, ovvero la stessa mappa con i dati estratti dai file SRTM.

\begin{bash}
osmosis -plugin org.srtmplugin.osm.osmosis.SrtmPlugin_loader --read-xml "\$CITY_NAME".osm --write-srtm locDir="\$SRTM_HOME" locOnly=true repExisting=false --write-xml "\$CITY_NAME"_srtm.osm
\end{bash}

%

Da tenere in considerazione il fatto che il comando mostrato è incluso nello script di generazione automatica da me creato al fine di velocizzare la configurazione dello scenario.

\subsection{Generazione XML di SUMO}

SUMO necessita di file di configurazione in XML che descrivono la rete stradale, i poligoni dei palazzi e i percorsi di ogni singolo veicolo. Siccome ognuno di questi file, per essere generato, richiede un apposito comando il quale a sua volta richiede vari parametri, ho creato uno script che data la mappa di una città in formato Open Street Map esegue tutte le operazioni necessarie.

Verranno comunque analizzati tutti i comandi singolarmente in modo d aavere una panoramica sulle scelte implementative.

\subsubsection{La rete Stradale (.net.xml)}

Il file della rete stradale viene generato attraverso il tool \code{netconvert} direttamente dalla mappa di Open Stree Map. Oltre al file \code{.osm} è necessario anche un file di supporto che istruisca SUMO sui vincoli e i limiti di velocità delle strade importate. Noi ne utilizziamo uno creato ad hoc per il traffico tedesco. 

Qui sotto ne riporto un frammento a puro titolo esemplificativo, il file intero si trova in \code{simulator/veins-2.1/examples/veins/bologna/osm-urban-de.typ.xml}

\begin{xml}
<types xmlns:xsi="http://www.w3.org/2001/XMLSchema-instance">
  <type id="highway.motorway" priority="13" numLanes="2" speed="41.667"
                oneway="true" disallow="bicycle pedestrian"/>
  <type id="highway.motorway_link" priority="8" numLanes="1" speed="13.889"/>
  <type id="highway.trunk" priority="12" numLanes="2" speed="13.889"/>
  <type id="highway.trunk_link" priority="8" numLanes="1" speed="13.889"/>
  <type id="highway.primary" priority="11" numLanes="2" speed="13.889"/>
  <type id="highway.primary_link" priority="8" numLanes="1" speed="13.889"/>
  <type id="highway.secondary" priority="10" numLanes="2" speed="13.889"/>
  ....
</types> 
\end{xml}

Di seguito passiamo ad un analisi dettagliata di tutti i parametri passati a \code{netconvert}:

\begin{itemize}
	\item \textbf{-{}-type-files}: Specifica il file che contiene i vincoli e i limiti, quello citato sopra.
	\item \textbf{-{}-ramps.guess}: Prova a capire dove sono le rampe e ad eseguirne l'importazione
	\item \textbf{-{}-remove-edges.by-vclass}: Siccome Open Street Map include un infinità informazioni del tutto inutili al nostro fine (ferrovie, piste ciclabili, aree pedonali ecc..) con questo parametro si indicano le classi da non importare (\code{bicycle,pedestrian...})
	\item \textbf{-{}-geometry.remove}:
	\item \textbf{-{}-remove-edges.isolated}:
	\item \textbf{-{}-tls.join}:
	\item \textbf{-{}-osm-files}:
	\item \textbf{-{}-output.street-names}:
	\item \textbf{-{}-output.original-names}:
	\item \textbf{-{}-output-file}:
\end{itemize}

Siccome Open Street Map include molte informazioni che sono del tutto inutili al nostro fine (ferrovie, piste ciclabili, aree pedonali ecc..) bisogna istruire \code{netconvert} affinché le escluda dall'importazione. 

%\begin{bash}
%netconvert --type-files osm-urban-de.typ.xml --ramps.guess --remove-edges.by-vclass hov,taxi,bus,delivery,transport,lightrail,cityrail,rail_slow, rail_fast,motorcycle,bicycle,pedestrian --geometry.remove --remove-edges.isolated true --tls.join --osm-files "$MAP_FILE" --output.street-names --output.original-names --output-file "$CITY_NAME".net.xml
%\end{bash}


	
	\clear
	
	\chapter{Conclusioni}

Al termine di questa esperienza durata oltre un anno posso asserire di aver portate un grande contributo al progetto Internet of Energy su diversi fronti. Partendo da un progetto già esistente, ma in fase estremamente embrionale, ho creato una piattaforma solida e modulare. Il principio di base che ha guidato lo sviluppo è stato rendere l'architettura portabile e semplice da capire al fine di permettere a chiunque di partecipare attivamente al progetto. Difatti quattro sono stati i progetti di entità minore sviluppati in seno a questo progetto, personalmente ho seguito ognuno di essi, ed è stata una notevole soddisfazione vedere i risultati che ne sono conseguiti.

La modularità è stata ottenuta applicando i noti concetti di ingegneria del software che hanno portato ad avere un software più facile da capire, e quindi tramandare, e mantenere. 

Ho automatizzato diverse fasi del progetto, infatti bisogna considerare che quando ho preso a mano il progetto lo scenario simulabile era uno solo e discostarsi da esso risultava notevolmente difficile. Oggi in meno di mezz'ora si è in grado di creare uno scenario funzionante per uno scenario urbano completamente diverso da Bologna. 

In primo luogo l'applicazione mobile, aspetto che mi ha permesso di accedere al progetto, è cresciuta notevolmente... bla bla bla...

In secondo luogo il Servizio Cittadino in 


Infine la piattaforma di simulazione ha raggiunto un notevole grado di realisticità e stabilità che hanno portato ad aderire appieno ai requisiti del progetto IoE. La maturità e la genericità della piattaforma sviluppata è tale che ha attirato l'interesse per un altro grande progetto europeo, ARROWHEAD. 








	  
	\appendix%tutti i capitoli da qui in avanti sono considerati appendici
	  
	\chapter{Installazione Ambiente}

\section{Installazione}

Per far interagire tutti gli elementi necessari alla simulazione è necessario installare numerosi framework e librerie. In questa sezione verrà data una guida il più esaustiva possibile per installare e configurare un ambiente funzionante. Verranno inoltre forniti i link specifici per l'installazione di ogni componente qualora insorgano delle problematiche.

Il procedimento di installazione è testato e funzionante su Debian \emph{7} Wheezy (con versioni precedenti potrebbero esserci problemi con le versioni delle librerie) e Ubuntu dalla versione \emph{12.10} alla \emph{13.10}. È stato anche possibile completare l'installazione su MacOSX ma non essendomene occupato personalmente non posso assicurare nulla al riguardo.

\subsection{Installazioni preliminari}

Questi sono i pacchetti che vanno installati su Debian 7 al fine di installare tutti i componenti successivi. Non è sicuro che siano gli unici necessari. È probabile che lo stesso comando vada bene anche per Ubuntu.

\begin{bash}
sudo apt-get install bison flex build-essential zlib1g-dev tk8.4-dev blt-dev libxml2-dev libpcap0.8-dev autoconf automake libtool libxerces-c2-dev libproj-dev libproj0 libfox-1.6-dev libgdal1h libboost-dev
\end{bash}

\subsection{OMNeT++} 

AL momento di scrivere questo documento la versione usata per il progetto è la 4.4 ma in generale le versioni dalla 4.2 in su dovrebbero andare bene. Questo è il link per la versione 4.4 \url{http://www.omnetpp.org/omnetpp/cat_view/17-downloads/1-omnet-releases}. Dopo aver scaricato il tar.gz lo si estragga e si proceda con l'installazione:

\begin{bash}
./configure
make
bin/omnetpp
\end{bash}

\bigskip
\noindent
Durante l'installazione verrà detto di inserire alcune variabili d'ambiente nel file .bashrc non dimenticarsi di eseguire queste direttive.

In Ubuntu 13.10 si può assistere a un bug che determina la sparizione dei menu di OMNeT++, per risolverlo è necessario impostare la seguente variabile d'ambiente nel file  \code{$\sim$/.basrc}:

\begin{bash}
export UBUNTU_MENUPROXY=0
\end{bash}

\noindent
per maggiori informazioni guardare questa discussione su StackOverflow \url{http://stackoverflow.com/questions/19452390/eclipse-menus-dont-show-up-after-upgrading-to-ubuntu-13-10}

\subsection{SUMO}

Seppur SUMO sia disponibile tra i pacchetti di Debian/Ubuntu è necessario comunque scaricare i sorgenti tramite SVN di una versione successiva alla \emph{15340} e compilarli. Questo perchè la versione attualmente disponibile tramite il gestore di pacchetti, ovvero la \emph{0.19.0}, non supporta l'importazione nelle mappe (i file .net.xml) dei dati del profilo altimetrico, fondamentali per avere un modello di consumo energetico del veicolo realistico.

Quindi i comandi necessari, presupponendo di avere Subversion installato, sono:

\begin{bash}
svn co https://sumo.svn.sourceforge.net/svnroot/sumo/trunk/sumo
make -f Makefile.cvs
./configure
make
sudo make install
\end{bash}

Per una trattazione più completa dell'installazione rimando il sito ufficiale \url{http://sourceforge.net/apps/mediawiki/sumo/index.php?title=Installing/Linux_Build}

\subsection{SMART-M3}

La tecnologia Smart-M3 forisce la SIB, ovvero il database semantico usato per lo scambio di informazioni tra i vari componenti del sistema. Noi utilizzeremo nello specifico la RedSIB sviluppata da ARCES e basata su un progetto di Nokia
(Nokia C Smart M3).
La versione supportata dal nostro ambiente è la 0.9 ma anche le successive dovrebbero andare bene. Il link per il download è questo: \url{http://sourceforge.net/projects/smart-m3/files/Smart-M3-RedSIB_0.9/}. Una volta estratto il tar.gz al suo interno troveremo sia i sorgenti che i pacchetti per Debian. Nel caso si intenda compilare i sorgenti rimando alle istruzioni contenute all'interno del pacchetto. Qui ci limiteremo a installare i deb attraverso gli script forniti:

\begin{bash}
sudo ./install.sh     #per architetture x86
sudo ./install_x64.sh #per architetture amd64
\end{bash}

All'interno del pacchetto viene data la possibilità di utilizzare Virtuoso come database RDF ma, seppur probabilmente sia più performante, non lo utilizzeremo in quanto è una feature introdotta recentemente e quindi non abbastanza testata.

\subsection{KPI_Low}

La libreria KPI_Low è un API scritta in C che, attraverso il protocollo SSAP, permette di interfacciarsi alla SIB. È stata scritta da Jussi Kiljander, un ricercatore del VTT Technical Research Centre of Finland, e successivamente modificata da Federico Montori di UNIBO per aggiungervi il supporto alle query SPARQL. Io l'ho modificata al fine di rimuovere dei Memory Leak trovati grazie al tool Valgrind.
In quanto la versione della libreria non è quella originale è necessario usare la nostra versione che si trova nella cartella \code{kpi_low_mod} nella root del progetto.
Le KPI_Low necessitano della libreria SCEW per il parsing XML, la quale non si trova nei repository di Debian/Ubutnu,  è quindi necessario scaricarla dal seguente indirizzo \url{http://nongnu.askapache.com/scew/scew-1.1.3.tar.gz} e compilarla. Una volta scaricata estrarla e spostarsi nella cartella estratta:

\begin{bash}
./configure
make
sudo make install
\end{bash}

Adesso possiamo procedere con l'installazione delle KPI_Low, spostarsi dunque nella cartella \code{kpi\_low\_mod}:

\begin{bash}
./autogen.sh
./configure
make
sudo make install
\end{bash}

per istruzioni più dettagliate guardare il documento \code{kpi_low_mod/KPI_Low.pdf}

\subsection{Importare il progetto in OMNeT++}

Adesso che abbiamo predisposto l'ambiente possiamo procedere con l'importazione in OMNeT++ del simulatore e con la compilazione. Apriamo OMNeT++, se è il primo avvio ci chiederà che Workspace usare proponendocene uno predefinito, in tal caso noi scegliamo la cartella \code{simulator} all'interno della root del progetto. Probabilmente verrà chiesto anche se si vuole abilitare il supporto ai framework MiXiM e INET e se si vogliono importare i porgetti di esmpio, in entrambi i casi diciamo di no. Nel caso in cui il workspace fosse già impostato allora andiamo su \code{File -> Switch Workspace -> Other...} e selezioniamo la cartella \code{simulator} nella root del progetto proprio come sopra. Se a seguito della selezione del workspace \code{simulator} la scheda dei progetti rimane vuota allora andiamo su \code{File -> Import... -> General/Existing Project into Workspace -> Next} e come root directory scegliamo \code{simulator}, dovremmo vedere il progetto \code{veins-2.1} nel riquadro \code{Projects}, lo selezioniamo e clicchiamo su \code{Finish}.

A questo punto non rimane che compilare il progetto. La compilazione può avvenire in due modalità:

\begin{itemize}
	\item \textbf{gcc-debug}: Compila includendo le informazioni di debug rendendo possibile l'utilizzo di \code{gdb} per analizzare il funzionamento del programma. OMNeT++ mette a disposizione un front-end visuale per \code{gdb} che permette di inserire breakpoint nel sorgente ed eseguire l'avanzamento step a step. Inoltre permette di visualizzare il contenuto delle variabili durante l'esecuzione semplicemente semplicemente spostando il cursore sulla variabile interessata nel riquadro dei sorgenti. Queste funzionalità sono da prendere seriamente in considerazione qualora, a seguito di modifiche, la simulazione dovesse fallire.
	\item \textbf{gcc-release}: Compila non includendo le informazioni di debug e applicando le ottimizzazioni previste dal compilatore \code{gcc} con il flag \code{-O2}. Ovviamente questa configurazione è più performante della precedente e andrebbe usata quando, una volta ritenuto stabile il codice, si vogliono eseguire simulazioni batch.
\end{itemize}

Il cambio di modalità di compilazione si può effettuare tramite: \code{Tasto DX su veins-2.1 -> Build Configurations -> Set Active -> gcc-debug/gcc-release}.

I file che fanno parte del simulatore si trovano sotto la directory \code{simulator/veins-2.1/examples/veins}. 


\chapter{UniboGeoTools}\label{chap:unibo-geo-tools}

Libreria java sviluppata per motivi strani

	\clear

	\listoffigures%			produce l'indice delle figure
	\addcontentsline{toc}{part}{\small{Elenco delle figure}}%inserisco la voce ``Elenco delle figure'' che fa riferimento a questa parte nell'indice (come nuova parte)
	
	\clear
	
	\lstlistoflistings%		produce l'indice dei codici
	\addcontentsline{toc}{part}{\small{Elenco dei codici}}%inserisco la voce ``Elenco dei codici'' che fa riferimento a questa parte nell'indice (come nuova parte)
	
	\clear
		
	\chapter*{\refname}
	\printbibliography[heading=cartaceo,nottype=online]
	\printbibliography[heading=web,type=online]

  
\end{document}