\usepackage{listings} %box per codice sorgente
\usepackage{color} %colori
\usepackage{courier} %usa il curier in \ttfamily

% pacchetto che usa bergami!!
% permette ad esempio di usare \ttfamily (monospaced) nel codice e avere comunque alcune parole in grassetto (come loop nel codice arduino, ecc..)
\usepackage[scaled]{beramono}

\newcommand{\code}[1]{\texttt{#1}}

% per vedere ``Listato'' anziché ``Listing'' come didascalia sopra ai frammenti di codice
\addto\captionsitalian{\renewcommand{\lstlistingname}{Listato}}
% per vedere ``Listato'' anziché ``Listings'' come indice dei frammenti di codice
\addto\captionsitalian{\renewcommand{\lstlistlistingname}{Elenco dei Listati}}

%\DeclareCaptionFont{white}{\color{white}}
%\DeclareCaptionFormat{listing}{\colorbox[cmyk]{0.43, 0.35, 0.35,0.01}{\parbox{\dimexpr\textwidth-2\fboxsep\relax}{#1#2#3}}}
%\captionsetup[code]{format=listing,labelfont=white,textfont=white, singlelinecheck=false, margin=0pt, font={bf,footnotesize}}

\definecolor{my_gray}{RGB}{240,240,240}


\lstdefinelanguage{BASH} {
	otherkeywords={=, +, [, ], (, ), \{, \}, *},
    %morecomment=[l]{#},
    %morestring=[s]{"}{"},
    morekeywords={make,apt-get,sudo},
    breaklines=true,
    keywordstyle=\bfseries,
    stringstyle=\color{red},
    emphstyle=\color{black}\bfseries,
    commentstyle=\color{darkgreen}\slshape,
    alsoletter={-,/,.},
}

\lstnewenvironment{bash}[1][] {
	\lstset{language=BASH,
           basicstyle=\ttfamily\footnotesize,
           numbers=left,
           numberstyle=\tiny\color{gray},
           backgroundcolor=\color{lightgray},
           breaklines=true,
           breakatwhitespace=true,
           mathescape=false,
           morecomment=[l]{\#},
           stepnumber=1,
		   firstnumber=1,
           #1
          }
}{}

%XML
\definecolor{maroon}{rgb}{0.5,0,0}
\definecolor{darkgreen}{rgb}{0,0.5,0}
\lstdefinelanguage{XML} {
	basicstyle=\ttfamily,
	morestring=[s]{"}{"},
	morecomment=[s]{?}{?},
	morecomment=[s]{!--}{--},
	commentstyle=\color{darkgreen},
	moredelim=[s][\color{black}]{>}{<},
	moredelim=[s][\color{red}]{\ }{=},
	stringstyle=\color{blue},
	identifierstyle=\color{maroon}
}

\lstnewenvironment{xml}[1][] {
	\lstset{language=XML,
            basicstyle=\ttfamily\footnotesize,
            numbers=left,
            numberstyle=\tiny\color{gray},
            backgroundcolor=\color{lightgray},
            breaklines=true,
            breakatwhitespace=true,
            stepnumber=1,
			firstnumber=1,
			captionpos=bc,
           #1
          }
}{}

%%%%%%%%%%%%%%%%%%%%%%%%%% JAVA %%%%%%%%%%%%%%%%%%%%%%%%%%%%%%%%

\definecolor{javared}{rgb}{0.6,0,0} % for strings
\definecolor{javagreen}{rgb}{0.25,0.5,0.35} % comments
\definecolor{javapurple}{rgb}{0.5,0,0.35} % keywords
\definecolor{javadocblue}{rgb}{0.25,0.35,0.75} % javadoc

\lstnewenvironment{java}[1][] {
	\noindent\minipage{\linewidth}%
	\lstset{language=Java,
			basicstyle=\ttfamily\footnotesize,
			keywordstyle=\color{javapurple}\bfseries,
			stringstyle=\color{javared},
			commentstyle=\color{javagreen},
			morecomment=[s][\color{javadocblue}]{/**}{*/},
			showspaces=false,
			showstringspaces=false
            basicstyle=\ttfamily\footnotesize,
            numbers=left,
            tabsize=2,
            numberstyle=\tiny\color{black},
            backgroundcolor=\color{lightgray},
            breaklines=true,
            breakatwhitespace=true,
            stepnumber=1,
			firstnumber=1,
			captionpos=bc,
            #1
          }
	%\captionsetup{options=code}
}{\endminipage}

\lstnewenvironment{cpp}[1][] {
	\noindent\minipage{\linewidth}%
	\lstset{language=C++,
			basicstyle=\ttfamily\footnotesize,
			keywordstyle=\color{javapurple}\bfseries,
			stringstyle=\color{javared},
			commentstyle=\color{javagreen},
			morecomment=[s][\color{javadocblue}]{/**}{*/},
			showspaces=false,
			showstringspaces=false
            basicstyle=\ttfamily\footnotesize,
            numbers=left,
            tabsize=2,
            numberstyle=\tiny\color{black},
            backgroundcolor=\color{lightgray},
            breaklines=true,
            breakatwhitespace=true,
            stepnumber=1,
			firstnumber=1,
			captionpos=bc,
            #1
          }
	%\captionsetup{options=code}
}{\endminipage}

%http://jevopi.blogspot.it/2010/03/nicely-formatted-listings-in-latex-with.html